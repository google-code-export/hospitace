\chapter{Závěr}

Výstupem mé práce je aplikace pro evidenci hospitací. Aplikace je nasazená na serveru
http://kvalitavyuky.felk.cvut.cz tak, že ji lze použít v reálném provozu na ČVUT FEL. Samotná aplikace byla implementovaná na platformě Ruby on Rails, což je jeden z nejmodernějších frameworků pro vývoj webových aplikací. Aplikace používá dvě fakultní aplikace. Pro autentizaci používá aplikaci FELid, což je globální autentizační systém pro webové aplikace na fakultě FEl. Druhou aplikací je KOSapi, ta poskytuje aplikační rozhraní k přístupu dat v KOSu.

Pro vývoj aplikace byl použit iterační vývoj, kdy při každé iteraci byly zpracovány požadavky zadavatele a následně byli zdokumentovány. Snažil jsem se vytvořit funkční a uživatelsky přívětivou aplikaci. Funkčnost aplikace jsem demonstroval simulací probíhajících hospitací pro studijní program STM v LS 2011/2012. Bohužel se nepodařilo v čas aplikaci nasadit do reálného provozu.

Tato práce mi přinesla spoustu zkušenosti jak s novou technologií Ruby on Rail, kterou jsem do té doby nepoužil, tak i s použitím externích aplikací.

\section{Možnosti v pokračování práce}
Možností pokračování v práci je několik. V první řadě je potřeba nainstalovat a nastavit na serveru emailový server pro odesílání informačních emailových zpráv.

Lze také pokračovat ve vývoji dynamických formulářů. Do aplikace jsem totiž neimplementoval uživatelské prostředí pro nastavování formulářů. Dále by bylo dobré předělat funkční část do knihovny. Jako dalším krokem pro vylepšení aplikace bych provedl refaktoring celé aplikace, protože při vývoji jsem se postupně učil pracovat v Ruby on Rails.
