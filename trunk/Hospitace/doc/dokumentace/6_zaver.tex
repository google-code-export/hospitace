\chapter{Závěr}

\section{Zhodnocení splnění cílů}
Cílem bakalářské práce bylo vytvořit webovou aplikaci pro evidenci hospitací, která by navazovala  na již dříve vytvořený avšak nefunkční prototyp. Musí být vytvořena a doladěna tak, aby ji šlo použít v reálném provozu na ČVUT Fakultě elektrotechnické. Hlavním účelem vytvoření této aplikace byla snaha o snížení administrativní zátěže při správě hospitací. Aplikace má usnadnit práci při neustále se zvyšujícím počtu hospitací.

Podařilo se mi vytvořit aplikaci, která slouží k tomuto účelu a je implementována na platformě Ruby on Rails, což je jeden z nejmodernějších frameworků pro vývoj webových aplikací. Pro nasazení aplikace do reálného provozu používám dvě fakultní aplikace. Pro autentizaci je použita aplikace FELid, což je globální autentizační systém pro webové aplikace FEL. Druhou aplikací je KOSapi, ta poskytuje aplikační rozhraní k přístupu dat v KOSu.

Součástí práce bylo i vytvoření dvou rozsáhlých částí aplikace. V první části  jsou dynamické formuláře, které umožňují vytvářet formuláře s možností editace struktury. V druhé části aplikace je knihovna, která umožňuje generovat obsah informačních emailů. K vygenerování obsahu jsou využívány šablony a získaná data z hospitací. Výsledná aplikace je nasazena na serveru http://kvalitavyuky.felk.cvut.cz a připravena k použití v reálném provozu. 

\section{Osobní přínos}
Při řešení této práce jsem získal spoustu vědomostí a zkušeností, protože obsah práce byl obsáhlý a rozmanitý. Práce zahrnovala implementaci aplikace a její nasazení včetně instalace a nastavení serveru. Největším přínosem pro mne je poznání nového a zajímavého programovacího jazyka Ruby a moderního frameworku Ruby on Rails. Při tvorbě jsem narazil i na problémy s externími aplikacemi, které jsem musel řešit, což mě přimělo hlouběji se touto problematikou zabývat.

\section{Možnosti pokračování}
Možností pokračování v práci je několik. Vhodné by bylo vyřešit provizorní odesílání informačních emailových zpráv nainstalováním a nastavením emailového serveru. 

Lze také pokračovat ve vývoji dynamických formulářů. Jelikož jsem neimplementoval uživatelské prostředí pro jejich nastavování, je možné předělat jádro těchto formulářů do samostatné knihovny. 
V budoucnu bude účelné rozšířit aplikaci o podporu nové verze KOSapi, která je momentálně ve vývoji. Tato nová verze již bude obsahovat data z Bílé knihy\footnote{je to dokument, který obsahuje informace o organizací studijních plánů} a tudíž bude možné aplikaci rozšířit o podporu studijních programů i s obory.

