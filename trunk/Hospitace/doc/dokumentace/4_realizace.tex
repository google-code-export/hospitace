\chapter{Realizace}
Kvůli postupnému nasazování aplikace pro letní semestr 2011/2012 byl zvolený iterační vývoj aplikace. V této kapitole budu popisovat cil, postup a výstup jednotlivých iterací.

\section{První iterace}
\subsection{Cíl}
První iterace měla za cíl vytvořit základní architekturu aplikace s připojením na KOSapi. K tomu vytvořit funkční aplikaci pro naplánování hospitací.

\subsection{Postup}
\subsubsection{Datová vrstva}
Protože aplikace využívá dva datové zdroje KOSapi a databázi aplikace, bylo potřeba nejdřív vyřešit jak se připojit ke KOSapi. V této části vycházím z prototypu aplikace pro správu hospitací. Poté jsem vytvořil modely tak aby umožnily komunikaci mezi KOSapi tak i databázi.

\subsubsection{Autentizace a autorizace}
Pro autentizaci v této fázi vývoje jsem zatím neimplementoval autentizaci prostřednictvím FElid. Dočasně jsem pro tento účel použil modul authlogic.

Pro autorizaci používám modul CanCan. CanCan je modul pro Ruby on Rails, které omezuje, k jakým zdrojům má daný uživatel povolený přístup. Všechna oprávnění jsou definovány v jednom místě (třída Ability). Umí filtrovat controllery, views i databázové dotazy.

\subsubsection{Uživatelské prostředí}
Pro vytvoření uživatelského prostředí jsem použil již existující knihovnu Bootstrap od Twitteru. Použil jsem tuto knihovnu protože obsahuje jak kompletní CSS tak i Javascript a je open-source. Další výhodou je popularita uživatelského prostředí.

Pro usnadnění implementace Bootstrapu do aplikace jsem zprovoznil dva moduly. Prvním je SimpleForm ten zjednoduší vytváření formulářů, tak že si předem nadefinujeme jejich styl při kompletaci. A druhý modul WillPaginate ten slouží k implementaci stránkování dat.

\subsection{Výstup} 
Po ukončení této iterace jsem měl nasaditelnou část aplikace pro plánování hospitací.
