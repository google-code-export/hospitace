\chapter{Popis problému, specifikace cíle}
\section{Motivace}
Jak jsem uvedl v úvodu, tak v současné době probíhá jakákoliv administrativní činnost okolo hospitací převážně pomocí emailové komunikace. Kterou zajišťuje garantem studijního oboru přidělený administrátor kontroly výuky, který je dále  uváděn jako administrátor, nebo administrátor hospitací. 

Jeho úkolem je starat se o plánování hospitací a vystavování dokumentů na stránkách rady studijního programu \cite{kvalitavyukyweb}. Při naplánování hospitace musí administrátor ručně obeslat emailem informaci o naplánované hospitaci všem zainteresovaným osobám. To jsou hospitovaní, hospitující, přednášející a garanti příslušného předmětu. Po provedení hospitace je potřeba shromáždit a vystavit veškeré dokumenty na webových stránkách Rady programu. Administrátor musí hlídat tok dokumentů a rozesílat vzniklé dokumenty mezi účastníky hospitace.

Tento systém sice funguje, ale je administrativně a časově náročný jak pro administrátora hospitací, který se stará o komunikaci mezi účastníky, tak i pro zúčastněné strany hospitace. Proto byl podán požadavek na vytvoření systému pro evidenci hospitací, který zautomatizuje vnitřní procesy pro správu hospitací.

\section{Cíle práce}
Hlavním cílem mé práce je rozšířit prototyp aplikace pro evidenci hospitací, tak aby bylo možné ji nasadit do reálného provozu na FEL ČVUT. V průběhu letního semestru 2011/2012 se bude postupně demonstrovat její funkčnost na realizovaných hospitacích v daném semestru.

\section{Rešerše}
Hospitace jak už jsem nastínil v motivaci je zavedený vnitřní proces kontroly kvality výuky na ČVUT FEL, proto čerpám informace pro tvorbu této práce z dvou hlavních zdrojů. Prvním zdrojem je dokument Postupy pro kontrolu kvality výuky \cite{postupy}, který definuje jak se mají hospitace provádět. A druhým zdrojem je prototyp aplikace z rozpracované bakalářské práce Daniela Krężeloka Návrh a implementace systému pro správu hospitací \cite{prototyp_documentace}.

\subsection{Prototyp}

Jako referenční řešení práce jsem obdržel prototyp aplikace napsaný v Ruby on Rails. Ten se skládá z dokumentační části a z rozpracované webové aplikace. Dokumentační část prototypu obsahuje analýzu a návrh aplikace. Hlavně na začátku mi velmi pomohl pochopit problematiku hospitací a velmi užitečné informace pro samotný vývoj mé aplikace. Horší to bylo se samotnou aplikací. Verze programu kterou jsou získal nebyla dodělaná jelikož se mi ji nepodařilo zprovoznit tak aby byla funkční. Po hlubším zkoumání a upravování zdrojových kódu jsem objevil co aplikace umí. Bylo na implementované plánování hospitací a k nim šablony hodnotících formulářů. Další co prototyp umí je získat data přímo z KOSapi \ref{kosapi}. 

Důvod proč navazuji na práci Daniel Krężelok je ten že se rozhodl kompletně přejít na jinou technologii než je Ruby on Rails. Nakonec po hlubší analýze prototypu programu jsem rozhodl převzít z prototypu aplikace pouze připojení ke KOSapi a zbytek udělat od základů nový. 

\subsection{Postupy pro kontrolu kvality výuky}
V této části kapitoly popisuji hlavní procesy při vykonávání hospitací. Tyto informace čerpám z již zmíněného dokumentu Postupy pro kontrolu kvality výuky \cite{postupy}.

\subsubsection{Administrátor hospitací}
Administrátorem hospitací zajišťuje plánování a řízení hospitací pro studijní obor ke kterému je přiřazen. Administrátora hospitací je jedna osoba určená garantem studijního oboru.

\subsubsection{Plánování hospitací}
Pro naplánování hospitace musí administrátor hospitací definovat předmět,vyučovací hodinu a datum hospitace. Hospitace může probíhat jak na přednášky tak i na cvičení. Je potřeba i přiřadit hospitující\footnote{typicky je to dvojce pedagogů jeden zkušený a druhý začínající}, kteří provedou kontrolní návštěvu ve výuce.

Administrátor musí také určit typ hospitace. Existují tři druhy z hlediska vyučujícího:

\begin{itemize}
\item Předem ohlášené na konkrétní datum - u tohoto typu hospitace jsou veřejné všechny informace o naplánované hospitaci. Proto je potřeba, aby administrátor naplánoval hospitaci s předstihem a informoval o tom vyučujícího.
\item Předem ohlášené bez konkrétního termínu - tento typ hospitace na rozdíl od předešlého typu nemusí mít předem pevně stanovený datum hospitace. Hospitovaný  ví o hospitaci, ale neví kdy bude.
\item Předem neohlášené - u tohoto typu hospitace je předem informován pouze garant, zástupce garanta a administrátor výuky. 

\end{itemize}

\subsubsection{Provedení hospitace}
Kontrolní návštěva předmětu se provede dle naplánování hospitace a výstupem návštěvy z hodiny je dokument Hodnocení výuky, který dokumentuje průběh hodiny a její hodnocení. 

\subsubsection{Hodnocení výuky}
Po hospitaci se následuje hodnotící část. Ta se skládá ze čtyř druhů dokumentů, které hodnotí proběhlou výuku. Všechny dokumenty musí být předány administrátorovi hospitací. Ten je vystavení na privátní části webových stránek Rady programu \cite{kvalitavyukyweb}. Po sepsání a vystavení všech dokumentů je hospitace ukončená. 

\begin{itemize}
\item[A] Hodnocení výuky při hospitaci - ten dokument  je písemný výstup z hospitace, který slouží k popisu průběhu výuky. Skládá se z dokumentační části a z hodnotící části. Tento formulář musí vyplňují všichni hospitující.
\item[B] Slovní hodnocení hospitační návštěvy hospitujícím(mi) - tento dokument sepíše po hospitaci jeden z hospitujících a účelem tohoto dokumentu je slovně zhodnotit výuku. 
\item[C] Stanovisko hodnoceného učitele k názorům hospitujícího - Tímto dokumentem se může hospitovaný, garant předmětu a vedoucí katedry vyjádřit k slovnímu hodnocení. Proto tento dokument lze sepsat až po vystavení formuláře B.
\item[D] Závěrečné shrnutí hospitujícím - je poslední dokument a nejdůležitějším dokumentem. Sepíše ho jeden z hospitujících. Dokument slouží jako výstup z hospitace a obsahuje klady, zápory, navržená opatření a závěr. Tento dokument je veřejný proto je potřeba ho vytavit i na volně přístupných stránkách.
\end{itemize}