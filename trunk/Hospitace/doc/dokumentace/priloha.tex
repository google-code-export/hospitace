\chapter{UML diagramy a obrázky}
\section{Use cases}
\label{use_cases}

\begin{figure}[H]
\begin{center}
\includegraphics[width=12cm]{figures/actor_base}
\caption{Use case - nepřihlášený uživatel}
\label{fig:actor_base}
\end{center}
\end{figure}

\begin{figure}[H]
\begin{center}
\includegraphics[width=12cm]{figures/actor_logged}
\caption{Use case - přihlášený uživatel}
\label{fig:actor_logged}
\end{center}
\end{figure}

\begin{figure}[H]
\begin{center}
\includegraphics[width=12cm]{figures/actor_observed}
\caption{Use case - hospitovaný}
\label{fig:actor_observed}
\end{center}
\end{figure}

\begin{figure}[H]
\begin{center}
\includegraphics[width=12cm]{figures/actor_observer}
\caption{Use case - hospitující}
\label{fig:actor_observer}
\end{center}
\end{figure}

\begin{figure}[H]
\begin{center}
\includegraphics[width=12cm]{figures/actor_root}
\caption{Use case - administrátor}
\label{fig:actor_root}
\end{center}
\end{figure}

\begin{figure}[H]
\begin{center}
\includegraphics[width=14cm]{figures/actor_admin}
\caption{Use case - administrátor hospitací}
\label{fig:actor_admin}
\end{center}
\end{figure}


\chapter{Dynamické formuláře}
\label{sec:forms}
\begin{table}[h]
\begin{center}
\begin{tabular}{|l|c|l|}

\hline
\textbf{Typ} & \textbf{Návratová hodnota} & \textbf{Popis} \\ \hline
label &  & textový popisek \\\hline
integer & číslo & vstupní element pro čísla \\ \hline
text & text & formulář pro psaní textů \\\hline
text/file & text & formulář pro psaní textů s možností \\ & &  nahrání souboru s naskenovaným formulářem \\\hline
ranking\_table &  & tabulka pro hodnocení, může obsahovat několik \\ & & elementů typu \textit{ranking} \\\hline
column\_table & & tabulka do které se vkládají jiné elementy \\ & &  po sloupcích \\\hline
ranking & [A,B,C,D,E,F] & vstupní element pro zadávaní známkování  \\ & &   od A do F \\\hline
ranking\_scale & & tabulka s hodnotící stupnicí \\\hline
note & text & vstupní element pro napsání textové poznámky \\\hline

\end{tabular}
\caption{Seznam podporovaných typů elementů}
\label{tab:elements}
\end{center}
\end{table}

\begin{figure}[H]
\begin{center}
\includegraphics[width=14cm]{figures/form_A}
\caption{Hodnotící formulář A}
\label{fig:form_a}
\end{center}
\end{figure}

%*****************************************************************************
\chapter{Instalační a uživatelská příručka}
Instalační příručka je napsaná pro zprovoznění aplikace v prostředí development. Návod je napsaný pro operační systém Ubuntu 11.10. Pro zprovoznění aplikace je potřeba nainstalovat tyto programy:
\begin{itemize}
\item Ruby 1.9.3
\item RubyGems 1.8.13
\item Ruby on Rail 3.2.1
\item Databázový systém Mysql 5.5 nebo SQLite 3
\item Git
\end{itemize}

\section{Instalace}
Nejprve je potřeba nainstalovat platformu Ruby do systému. S ruby je potřeba nainstalovat i balík build-essential a git-core kvůli  knihovnám, které jsou potřeba pro zprovoznění aplikace. 

\begin{verbatim}
sudo apt-get install ruby1.9.3-full build-essential git-core
\end{verbatim}

Je potřeba nainstalovat i balíčkovací systém RubyGems a aktualizovat jeho verzi na poslední podporovanou.
\begin{verbatim}
sudo apt-get install rubygems1.8
sudo gem update --system
\end{verbatim}

V poslední fázi je potřeba nainstalovat Ruby on Rails pomocí balíčkovacího programu rubygems.

\begin{verbatim}
sudo gem install rails
\end{verbatim}

Po nainstalovaní Ruby on Rails je potřeba doinstalovat všech závislosti aplikace. Nejprve je potřeba někam do systémů zkopírovat aplikaci třeba do domovského adresáře. Ve všech dalších krocích je potřeba abychom se nacházeli v adresáři aplikace. Druhý příkaz slouží k instalaci závislostí aplikace.

\begin{verbatim}
cd ~/Hospitace
bundle install
\end{verbatim}  

\section{Konfigurace a příprava aplikace}
Před spuštěním aplikace je potřeba nastavit přístupové údaje k databázi. Nastavení databáze se nachází v souboru config/database.yml, který se nalézá v aplikaci. Pro zavedení databáze pak stačí jen spustit příkazy:

\begin{verbatim}
rake db:create
rake db:migrate
\end{verbatim}  

Teď už je vytvořená databáze, ale je ji potřeba naplnit daty z KOSapi. Pro tento účel slouží příkaz:

\begin{verbatim}
rake import
\end{verbatim}

\section{Spuštění aplikace}
Pro spuštění aplikace použijte webový server Webrick, který je součástí Ruby on Rails. Po spuštění bude aplikace běžet na portu 3000.
Příkaz pro spuštění:
 
\begin{verbatim}
rails server
\end{verbatim}

%*****************************************************************************
\chapter{Obsah přiloženého CD}
\begin{figure}[h]
	\dirtree{%
		.1 /.
		.2 readme.txt\DTcomment{stručný popis obsahu CD}.
		.2 src/\DTcomment{zdrojové kódy aplikace}.
		.2 prototyp/\DTcomment{prototyp aplikace}.
		.3 src/\DTcomment{zdrojové kódy prototypu}.
		.3 BP.pdf\DTcomment{text prototypu ve formátu PDF}.
		.2 thesis/\DTcomment{zdrojová forma práce ve formátu
\LaTeX{}}.
 		.3 figures/\DTcomment{obrázky pro text práce}.
		.2 text/\DTcomment{zdrojová forma práce ve formátu}.
		.3 thesis.pdf\DTcomment{text práce ve formátu PDF}.
	}
	\caption{Obsah CD}
\label{tree:obsah_cd}
\end{figure}
%$ tree . >tree.txt
