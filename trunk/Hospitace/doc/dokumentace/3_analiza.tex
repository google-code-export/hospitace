\chapter{Analýza a návrh řešení}
Cílem této kapitoly je vyhodnotit funkční a nefunkční požadavky na aplikaci. Definuji zde doménový model a případy užití s business procesy pro vytvoření hospitace. 

\section{Požadavky}
Požadavky na systém se dělí na dvě sekce funkčních a nefunkčních požadavky. Pro definování těchto požadavků vycházejí ze zadání práce. Tyto požadavky definují návrh a pozdější implementaci systému a definují meze systému.
\subsection{Obecné požadavky}
Obecné požadavky jsou požadavky, které se netýkají funkčnosti ale celkového návrhu a použitých technologií.
\begin{enumerate}
\item Systém bude postaven na webovém frameworku Ruby on Rails.
\item Systém bude používat databázi MySQL.
\item Systém bude webovou aplikací.
\item Serverová část systému poběží na aplikačním serveru Apache2.
\item Systém bude používat webovou službou KOSapi.
\item Systém bude pro autentizaci používat FELid.
\end{enumerate}

\subsection{Funkční požadavky}
Tato sekce se zabývá požadavky na funkčnost systému. Systém umožní:
\begin{enumerate}
\item Systém umožní spravovat uživatele.
\item Systém umožní průběžné plánování hospitaci.
\item Systém umožní přiřazení hospitujícího k hospitaci.
\item Systém umožní vyhledávat předměty z KOSapi.
\item Systém umožní vyhledávat osoby z KOSapi.
\end{enumerate}

\section{Uživatelské role}
Uživatelské role v aplikaci jsou odvozený z modelu aktérů systému u každého uživatele je diagram případu užití. Rozložení rolí v systému je takové, že uživatel systému se může nacházet v několika rolích. V systému jsou následující role:
\subsection{Nepřihlášený uživatel}
Nepřihlášený uživatel Nepřihlášený uživatel je host naší aplikace, má nejmenší pravomoc v systému. V tomto stavu je každý uživatel, který se doposud nepřihlásil do. Tento uživatel má právo pouze si prohlížet veřejné naplánované hospitace.
\subsection{Přihlášený uživatel}
Přihlášený uživatel je uživatel systému, který se již přihlásil do systému. Tento uživatel se umí pouze na rozdíl od nepřihlášeného uživatele pouze odhlásit do systému.
\subsection{Hospitovaný}
Hospitovaný je role pro přihlášeného uživatel v systému. Tento uživatel je vyučující, který vyučuje předmět na němž je naplánovaná hospitace. Tato role přidává možnost přístupu k naplánované hospitaci. U těchto hospitací má právo na zobrazení informací, zobrazení hodnocení a sepsat stanovisko k názorům hospitujícího.
\subsection{Hospitující}
Hospitující je role pro přihlášeného uživatele v systému. Je uživatel přidělený k vykonání hospitace. U přidělených hospitací má uživatel právo na zobrazení informací a možnost vytvořit hodnocení z vykonané hospitace.
\subsection{Administrátor hospitací}
Administrátor hospitací je přihlášený uživatel, který má právo spravovat hospitace a k nim přidělovat hospitující. Dále má právo přidělovat uživatelům role administrátor hospitací a hospitující.
\subsection{Administrátor}
Administrátor je uživatel který má nejvyšší pravomoc v systému. Má pravomoc spravovat všechny části aplikace. 

\section{Doménový model}
Doménový model reprezentuje klíčové domény systému a jejich vztahy mezi sebou. V diagramu je vidět propojení domén s doménami v KOSapi.
\subsection{Domény v KOSapi}
\begin{itemize}
\item Osoba uživatelé v kosu
\item Semestr semestry 
\item Předmět pře
\item Instance předmětu in
\item Paralelka par
\item Místnost místnost
\end{itemize}

\subsection{Domény aplikace}
\begin{itemize}
\item Uživatel
\item Hospitace
\item Poznámka
\item Příloha
\item Formulář
\item Hodnota
\item Typ formuláře
\item Položka
\end{itemize}

\section{Životní cyklus hospitace}

\section{Technologie}



