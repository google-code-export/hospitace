\chapter{Analýza}
Tato kapitola pojednává o analýze a návrhu vhodného řešení aplikace. Výstupem této analýzy jsou funkční a obecné požadavky. Dále návrh a popis domén a nejdůležitější případy užití s aktéry. 

\section{Požadavky}
Požadavky na systém se dělí na dvě sekce: obecné a funkční požadavky. Pro definování těchto požadavků jsem vycházel z oficiálního zadání práce tak i z prototypu aplikace, protože mi přesně definuje návrh aplikace a pro implementaci systému i potřebné technologie.

\subsection{Obecné požadavky}
Obecné požadavky se netýkají funkčnosti, ale celkového návrhu a použitých technologií.
\begin{enumerate}
\item Systém bude postaven na webovém frameworku Ruby on Rails.
\item Systém bude webovou aplikací.
\item Systém bude používat webovou službu KOSapi.
\item Systém bude pro autentizaci používat FELid.
\end{enumerate}

\subsection{Funkční požadavky}
Tato sekce se zabývá požadavky na funkčnost systému.
\begin{enumerate}
\item Systém umožní spravovat uživatele.
\item Systém umožní průběžné plánování hospitaci.
\item Systém umožní hospitujícímu i hospitovanému prohlížet hospitace \cite{prototyp_documentace}.
\item Systém umožní vystavit závěrečné hodnocení na veřejné části aplikace \cite{prototyp_documentace}. 
\item Systém umožní hospitovanému sepsat stanoviska k názorům hospitujícího \cite{prototyp_documentace}.
\item Systém umožní hospitujícímu nahrát naskenovaný dokument hodnocení výuky \cite{prototyp_documentace}.
\item Systém umožní hospitujícímu napsat slovní hodnocení z výuky \cite{prototyp_documentace}.
\item Systém umožní hospitujícímu napsat závěrečné shrnutí hospitace \cite{prototyp_documentace}.
\item Systém bude odesílat emailem zprávy o vyplnění hodnotícího dokumentu příslušným osobám \cite{prototyp_documentace}.
\item Systém umožní vyhledávat předměty z KOSapi.
\item Systém umožní vyhledávat osoby z KOSapi.
\item Systém umožní upravovat strukturu hodnotících dokumentů.
\item Systém umožní spravovat emailové šablony k hodnotících dokumentů.
\item Systém umožní generovat emailové zprávy ze šablon.
\item Systém bude automaticky zálohovat databázi.
\end{enumerate}

\newpage 
\section{Uživatelské role}
V systému je celkem 7 uživatelských rolí. Definoval jsem tři základní uživatelské role, které jsou základem systému: nepřihlášený uživatel, přihlášený uživatel, administrátor hospitací a admin. Další dvě role hospitovaný a hospitující se přidělují v rámci jednotlivých hospitací. Na obrázku \ref{fig:actors} jsou vidět jednotlivý aktéři a jejich zobecnění. V další části této sekce rozeberu jednotlivé role a k ním případy užití.

\paragraph*{Use cases}
neboli případy užití je nástroj pro popsání chování jak by systém měl spolupracovat s koncovým uživatelem. Popisuje všechny způsoby jak uživatel komunikuje se systémem.

\begin{figure}[H]
\begin{center}
\includegraphics[width=10cm]{figures/Actors}
\caption{Aktéři}
\label{fig:actors}
\end{center}
\end{figure}

\subsection{Nepřihlášený uživatel}
Nepřihlášený uživatel je role pro hosty naší aplikace. V systému má ze všech rolí nejmenší pravomoc. V tomto stavu je každý uživatel, který se doposud nepřihlásil do systému.

\begin{figure}[H]
\begin{center}
\includegraphics[width=10cm]{figures/actor_base}
\caption{Use case - nepřihlášený uživatel}
\label{fig:actor_base}
\end{center}
\end{figure}

\subsection{Přihlášený uživatel}
Přihlášený uživatel vychází z role nepřihlášeného uživatele. Je to uživatel, který se do systému přihlásil. Jedná se o základní roli pro všechny další role, které ji rozšiřují.

\begin{figure}[H]
\begin{center}
\includegraphics[width=10cm]{figures/actor_logged}
\caption{Use case - přihlášený uživatel}
\label{fig:actor_logged}
\end{center}
\end{figure}


\subsection{Hospitovaný}
Hospitovaný je role pro přihlášeného uživatele v systému. Je přidělena pro každého vyučujícího, který vyučuje předmět, na němž byla naplánovaná hospitace a proběhla.

\begin{figure}[H]
\begin{center}
\includegraphics[width=10cm]{figures/actor_observed}
\caption{Use case - hospitovaný}
\label{fig:actor_observed}
\end{center}
\end{figure}

\subsection{Hospitující}
Hospitující je role pro přihlášeného uživatele v systému. Tato role se přiděluje automaticky z naplánovaných hospitací, nebo ji může přidělit administrátor hospitací.

\begin{figure}[H]
\begin{center}
\includegraphics[width=10cm]{figures/actor_observer}
\caption{Use case - hospitující}
\label{fig:actor_observer}
\end{center}
\end{figure}

\subsection{Administrátor hospitací}
Hlavním úkolem této role je plánovat hospitace na předměty a posléze je spravovat.

\begin{figure}[H]
\begin{center}
\includegraphics[width=10cm]{figures/actor_admin}
\caption{Use case - administrátor hospitací}
\label{fig:actor_admin}
\end{center}
\end{figure}

\subsection{Administrátor}
Administrátor je super uživatel, který má nejvyšší pravomoc v systému. Má přístup ke všem zdrojům aplikace a může aplikaci spravovat.

\begin{figure}[H]
\begin{center}
\includegraphics[width=10cm]{figures/actor_root}
\caption{Use case - administrátor}
\label{fig:actor_root}
\end{center}
\end{figure}

\newpage 
\section{Doménový model}
Doménový model na obrázku \ref{fig:domainmodel} reprezentuje entity v systému a jejich vzájemné vztahy. 

Popis domény jsem pro přehlednost rozdělil podle zdroje na dvě základní skupiny. V první skupině jsou domény, které jsem převzal ze struktury KOSapi a druhou skupinou jsou domény specifické pro moji aplikaci. 

\label{sec:domeny_kosapi} 
\subsection{Domény z KOSapi}
\begin{itemize}
\item Osoba - je osoba v KOSu. Každá osoba může být učitelem a studentem.
\item Semestr - semestr vyučovaný na FEL. 
\item Předmět - předměty vyučované na FEL.
\item Instance předmětu - jsou instance předmětu vypsané v konkrétním semestru.
\item Paralelka - je vypsaná rozvrhová paralelka pro instanci předmětu.
\item Místnost - místnost na FEL
\end{itemize}

\subsection{Domény aplikace}
\begin{itemize}
\item Hospitace - obsahuje informace o naplánování hospitace. 
\item Poznámka - je textová poznámka k plánování hospitace.
\item Hodnocení - reprezentuje informace z proběhlé hospitace hospitace, jsou to informace o datu vykonání hospitace, hospitujícím, předmětu a garantovi.
\item Příloha - je připojený datový soubor k hodnocení hospitace.
\item Šablona formuláře - šablona pro tvorbu formulářů. Definuje vlastnosti jakým se budou vytvářet hodnotící formuláře.
\item Položka - položka reprezentuje jednotlivé segmenty formuláře. Tyto segmenty pak v celku definují strukturu formuláře.
\item Formulář - vyplněný hodnotící formulář.
\item Hodnota - je hodnota z vyplněného formuláře. Ta se ukládá z položky formuláře.
\item Šablona emailu - šablona emailu ze které se budou generovat emailové zprávy.
\end{itemize}

\begin{figure}[p]
\begin{center}
\includegraphics[width=14cm]{figures/DomainModel2}
\caption{Doménový model}
\label{fig:domainmodel}
\end{center}
\end{figure}



\section{Životní cyklus hospitace}
Cílem této části analýzy je popsat životní cyklus, kterým hospitace prochází. 

\subsection{Vytvoření}
Životní cyklus hospitace začíná jejím vytvořením. Toto zajišťuje administrátor hospitací, který založí hospitaci a definuje semestr, kdy se má hospitace uskutečnit, a předmět vyučovaný na fakultě. Při vytváření hospitace se určí typ hospitace a tím i její způsob zviditelnění, pro ostatní aktéry v aplikaci.

\subsection{Naplánování}
Při plánování je také hlavním aktérem administrátor hospitace. V této části životního cyklu administrátor určí hospitovanou paralelku předmětu a datum,  kdy se hospitace uskuteční.  

Administrátor také v této fázi přidělí hospitující z řad pedagogů určených k vykonání hospitace.  
 
\subsection{Hodnocení}
Poté, co proběhla kontrola hospitace, začíná nová fáze, ve které se hodnotí vyučování. Do této fáze už nezasahuje administrátor hospitace, ale přicházejí na scénu dva jiní aktéři: hospitovaný a hospitující.

V první fázi musí hospitující vyplnit, nebo nahrát naskenovaný formulář pro Hodnocení výuky při hospitaci. Tento formulář slouží k dokumentaci průběhu hospitace.

V druhé fázi jeden z hospitujících sepíše slovní hodnocení hospitační návštěvy.

Ve třetí fázi může hospitovaný do dvou dnů vyplnit stanovisko hodnoceného k názorům hospitujícího.  

V poslední fázi jeden z hospitujících sepíše poslední formulář Závěrečné shrnutí. Po vyplnění tohoto formuláře se hospitace stává ukončenou a tím končí její životní cyklu.
 
\subsection{Ukončená}
Po sepsáním posledního hodnotícího dokumentu se hospitace dostane do fáze ukončená. 

\begin{figure}[h]
\begin{center}
\includegraphics[width=14cm]{figures/hospitace}
\caption{Život hospitace}
\label{fig:hospitace}
\end{center}
\end{figure}
