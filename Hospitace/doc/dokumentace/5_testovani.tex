\chapter{Testování}

Cílem testování bylo otestovat reálnou funkčnost aplikace a odhalit chyby při návrhu a vývoji aplikace. Aplikaci jsem testoval dvěma způsoby: manuálně, kde jsem musel ručně procházet aplikaci a zkoušet její funkčnost, a automaticky pomocí testovacích nástrojů v Ruby on Rails. Aplikaci jsem testoval v průběhu celého vývoje. 

\section{Automatické testování}
Pro automatické testování jsem použil testovací nástroje v Ruby on Rails \cite{RoR_testing}. V Rails se testy dělí do tří kategorii: Unit, Functional a Integration. Jednotlivé kategorie testů, které jsem použil, popíši dále v textu. Testy jsem tvořil jen na kritických částech aplikace, kde byla velká pravděpodobnost vzniku chyb při úpravách funkcionality a návrhu aplikace.

\subsection{Testovací data}
Testovací nástroje potřebují pro svou funkčnost testovací data. Tyto data se nacházejí ve struktuře programu ve složce tests/fixtures. Zde jsou soubory s testovacími daty pro jednotlivé modely. Data jsou uložená ve formátu YAML\footnote{YAML je formát pro serializaci strukturovaných dat}. Před spuštěním testů se tato data
nahrají do testovací databáze. V mém případě je to lokální databáze SQLite. Díky oddělené databázi nemá testování vliv na produkční databázi.

\subsection{Unit testing}
Unit testy slouží k testování samostatných částí programů. V Rails se tyto testy používají hlavně pro testování funkčnosti modelů. Testuje se hlavně validace vstupů a perzistence dat. Tyto testy mi odhalily spoustu chyb, které vznikly při úpravách entit v aplikaci.

\subsection{Functional tests}
Tyto testy testují různé činnosti v jednotlivých controllerech aplikace. Controllery zpracovávají příchozí webové požadavky a nakonec odpovědí vyrendrovanou šablonou \cite{RoR_testing}.

\begin{list}{•}{\textbf{Tento typ testů testuje:}}
\item Byl webový požadavek úspěšný?
\item Byli jsme přesměrováni na správnou stránku?
\item Byli jsme úspěšně přihlášeni?
\item Byl objekt vložen do správné šablony?
\item Byla zobrazena správná hláška uživateli?
\end{list} 

\section{Manuální testování}
Manuálně jsem testoval ty části aplikace, u kterých se špatně vytvářejí  testy, nebo bylo potřeba použít lidský úsudek. Tyto vlastnosti splňuje testování uživatelského rozhraní a testování autorizace. Nejrozsáhlejší testování bylo vždy před koncem každé iterace. Důvodem bylo rozsáhlé testování reálné funkčnosti aplikace. Více v kapitole \ref{sec:sys_test}.

\section{Systémové testování}
\label{sec:sys_test}
Systémové testy jsou hlavní a nejdůležitější součástí testování. Smyslem tohoto testování
je ověřit funkčnost aplikace a zjistit, zda byly pokryty všechny požadavky dané zadavatelem. Nejčastěji tyto testy bývají prováděny  před nasazením do reálného provozu. 

Tento druh testů jsem prováděl vždy před  
koncem každé iterace, než jsem prezentoval svůj postup. Ujišťoval jsem se o stavu nové verze, zda není rozbitá a také jsem tím získal zpětnou vazbu o postupu vývoje. Při systémovém testování aplikace jsem netestoval všechny funkce aplikace, ale pouze ty nejdůležitější. Jsou to scénáře, které simulují běžnou práci uživatele. Při testování funkčnosti jsem simuloval průběh hospitací pomocí reálných dat získaných z probíhajících hospitací v letním semestru 2011/2012.

V příloze dokumentu \ref{test} jsou popsány jednotlivé testovací případy pro systémové testování. Výsledky těchto testů jsou znázorněny v tabulce \ref{tab:test}. V jednotlivých iteracích je znázorněno, zda aplikace testem prošla. Lze z ní také vyčíst i postupný vývoj aplikace. V poslední iteraci byly splněny všechny systémové testy, proto bylo možné aplikaci nasadit.


%Systémové testy - jsou hlavní částí testování z pohledu vývoje. Zde je ověřováno, že aplikace jako celek funguje správně.Testuje se, že správně plní úlohu, pro kterou byla vyvinuta, že vrací správné výstupy, že byly ošetřeny všechny nestandardní situace a v neposlední řadě, že byly pokryty všechny požadavky ze strany zákazníka. Systémové testy obvykle probíhají v několika kolech. Jsou hlášeny nalezené chyby, ty jsou opraveny a v následujících kolech retestovány.


% Musel jsem projít celou aplikaci a otestovat její funkčnost. 
%Při testování funkčnosti jsem simuloval průběh hospitací. K tomu jsem využíval reálná data z probíhajících hospitací v letním semestru 2011/2012.

\newpage
\begin{table}[h]
\begin{center}
\begin{tabular}{|c|c||c|c|c|c|}
\hline 
\textbf{ID} & \textbf{Test} & \textbf{1. iterace} & \textbf{2. iterace} & \textbf{3. iterace} & \textbf{4. iterace} \\
\hline 
1 & Přihlásit se &  &  &  & $\checkmark$ \\ 
\hline 
2 & Odhlásit se &  &  &  & $\checkmark$ \\ 
\hline 
3 & Seznam hospitací & $\checkmark$ & $\checkmark$ & $\checkmark$ & $\checkmark$ \\ 
\hline 
4 & Seznam hodnocení &  &  & $\checkmark$ & $\checkmark$ \\ 
\hline 
5 & Závěrečné hodnocení &  &  & $\checkmark$ & $\checkmark$ \\ 
\hline 
6 & Seznam hospituji & $\checkmark$ & $\checkmark$ & $\checkmark$ & $\checkmark$ \\ 
\hline 
7 & Zahájit hodnocení &  & $\checkmark$ & $\checkmark$ & $\checkmark$ \\ 
\hline 
8 & Formulář A &  & $\checkmark$ & $\checkmark$ & $\checkmark$ \\ 
\hline 
9 & Nahrát přílohu &  &  & $\checkmark$ & $\checkmark$ \\ 
\hline 
10 & Formulář B &  & $\checkmark$ & $\checkmark$ & $\checkmark$ \\ 
\hline 
11 & Formulář C &  & $\checkmark$ & $\checkmark$ & $\checkmark$ \\ 
\hline 
12 & Formulář D &  & $\checkmark$ & $\checkmark$ & $\checkmark$ \\ 
\hline 
13 & Odesílání emailů &  &  &  & $\checkmark$ \\ 
\hline 
14 & Seznam hospitován &  &  & $\checkmark$ & $\checkmark$ \\ 
\hline 
15 & Vytvořit hospitaci & $\checkmark$ & $\checkmark$ & $\checkmark$ & $\checkmark$ \\ 
\hline 
16 & Přiřadit hospitujícího & $\checkmark$ & $\checkmark$ & $\checkmark$ & $\checkmark$ \\ 
\hline 
17 & Naplánovat hospitaci & $\checkmark$ & $\checkmark$ & $\checkmark$ & $\checkmark$ \\ 
\hline 
18 & Napsat poznámku &  & $\checkmark$ & $\checkmark$ & $\checkmark$ \\ 
\hline 
19 & Přiřadit roli & $\checkmark$ & $\checkmark$ & $\checkmark$ & $\checkmark$ \\ 
\hline 
\end{tabular} 
\caption{Výsledky systémového testování}
\label{tab:test}
\end{center}
\end{table}

\paragraph{Struktura tabulky \ref{tab:test}:}
\subparagraph*{ID}
Identifikátor testovacího případu, ten lze podle ID dohledat v příloze dokumentu \ref{test}.
\subparagraph*{Test}
Hlavní cíl testu.
\subparagraph*{Iterace}
Obsahuje výsledky, zda byl test splněn v jednotlivých iterací.

