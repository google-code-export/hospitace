\chapter{Testování}

Cílem testování bylo otestovat reálnou funkčnost aplikace a odhalit chyby při vývoji aplikace. Aplikaci jsem testoval dvěma způsoby manuálně, kde jsem musel ručně procházet aplikaci a zkoušet její funkčnost a automaticky pomocí testovacích nástrojů v Ruby on Rails. Aplikace byla testována v průběhu celého vývoje.

\section{Automatické testování}
Automatické testy je důležité používat na místech, kde je velké riziko vzniku chyb. Chyby nejčastěji vznikají při přidávání, nebo úpravě funkcionalit v aplikaci. Pro automatické testování jsem použil testovací nástroje v Ruby on Rails \cite{RoR_testing}. V Ruby on Rails se testy dělí do tří kategorií  Unit, Functional a Integration.

\subsection{Testovací data}
Pro automatické testování je potřeba připravit testovací data. Testovací data jsou obsažená ve struktuře programu pod tests/fixtures. Zde se pak nacházejí soubory s testovacími daty pro jednotlivé modely ve formátu YAML.

\subsection{Unit testing}
Unit testy slouží k testování samostatných částí programů. V Rails se Unit testy používají hlavně pro testování funkčnosti modelů. Testuje se hlavně validace vstupů a perzistence dat.

\subsection{Functional tests}
Tyto testy testují různé činnosti v jednotlivých controllerech aplikace. Controllery zpracovávají příchozí webové requesty a nakonec vyrendrují view. Tento typ testů testuje:

\begin{list}{•}{}
\item Byl webový požadavek úspěšný?
\item Byli jsme přesměrováni na správnou stránku?
\item Byli jsme úspěšně přihlášeni?
\item Byl objekt vložen do správné šablony?
\item Byla zobrazena správná hláška uživateli?
\end{list} 

\subsection{Integration testing}
Integrační testy se používají k testování interakce mezi libovolným počtem kontrolorů. Tyto testy se používají k testování větších celků v rámci aplikace.

\section{Manuální testování}
Manuálně jsem testoval části aplikace u kterých se špatně vytvářejí automatické testy, nebo byl potřeba lidský úsudek\footnote{testování uživatelského prostředí}. Primárně jsem testoval reálnou funkčnost aplikace na konci každé iterace. Procházel jsem aplikaci podle scénářů užití a využíval jsem reálná data z probíhajících hospitací v letním semestru 2011/2012.