%% History:
% Pavel Tvrdik (26.12.2004)
%  + initial version for PhD Report
%
% Daniel Sykora (27.01.2005)
%
% Michal Valenta (3.12.2008)
% rada zmen ve formatovani (diky M. Duškovi, J. Holubovi a J. Žďárkovi)
% sjednoceni zdrojoveho kodu pro anglickou, ceskou, bakalarskou a diplomovou praci

% One-page layout: (proof-)reading on display
%%%% \documentclass[11pt,oneside,a4paper]{book}
% Two-page layout: final printing
\documentclass[11pt,twoside,a4paper]{book}   
%=-=-=-=-=-=-=-=-=-=-=-=--=%
% The user of this template may find useful to have an alternative to these 
% officially suggested packages:
\usepackage[czech, english]{babel}
\usepackage[T1]{fontenc} % pouzije EC fonty 
% pripadne pisete-li cesky, pak lze zkusit take:
% \usepackage[OT1]{fontenc} 
\usepackage[utf8]{inputenc}
%=-=-=-=-=-=-=-=-=-=-=-=--=%
% In case of problems with PDF fonts, one may try to uncomment this line:
%\usepackage{lmodern}
%=-=-=-=-=-=-=-=-=-=-=-=--=%
%=-=-=-=-=-=-=-=-=-=-=-=--=%
% Depending on your particular TeX distribution and version of conversion tools 
% (dvips/dvipdf/ps2pdf), some (advanced | desperate) users may prefer to use 
% different settings.
% Please uncomment the following style and use your CSLaTeX (cslatex/pdfcslatex) 
% to process your work. Note however, this file is in UTF-8 and a conversion to 
% your native encoding may be required. Some settings below depend on babel 
% macros and should also be modified. See \selectlanguage \iflanguage.
%\usepackage{czech}  %%%%%\usepackage[T1]{czech} %%%%[IL2] [T1] [OT1]
%=-=-=-=-=-=-=-=-=-=-=-=--=%

%%%%%%%%%%%%%%%%%%%%%%%%%%%%%%%%%%%%%%%
% Styles required in your work follow %
%%%%%%%%%%%%%%%%%%%%%%%%%%%%%%%%%%%%%%%
\usepackage{graphicx}
\usepackage{indentfirst} %1. odstavec jako v cestine.

\usepackage{k336_thesis_macros} % specialni makra pro formatovani DP a BP
 % muzete si vytvorit i sva vlastni v souboru k336_thesis_macros.sty
 % najdete  radu jednoduchych definic, ktere zde ani nejsou pouzity
 % napriklad: 
 % \newcommand{\bfig}{\begin{figure}\begin{center}}
 % \newcommand{\efig}{\end{center}\end{figure}}
 % umoznuje pouzit prikaz \bfig namisto \begin{figure}\begin{center} atd.


%%%%%%%%%%%%%%%%%%%%%%%%%%%%%%%%%%%%%
% Zvolte jednu z moznosti 
% Choose one of the following options
%%%%%%%%%%%%%%%%%%%%%%%%%%%%%%%%%%%%%
% \newcommand\TypeOfWork{Diplomová práce} \typeout{Diplomova prace}
% \newcommand\TypeOfWork{Master's Thesis}   \typeout{Master's Thesis} 
\newcommand\TypeOfWork{Bakalářská práce}  \typeout{Bakalarska prace}
% \newcommand\TypeOfWork{Bachelor's Project}  \typeout{Bachelor's Project}
%\newcommand\TypeOfWork{Semestrální práce}  \typeout{Semestralni prace}

%%%%%%%%%%%%%%%%%%%%%%%%%%%%%%%%%%%%%
% Zvolte jednu z moznosti 
% Choose one of the following options
%%%%%%%%%%%%%%%%%%%%%%%%%%%%%%%%%%%%%
% nabidky jsou z: http://www.fel.cvut.cz/cz/education/bk/prehled.html

%\newcommand\StudProgram{Elektrotechnika a informatika, dobíhající, Bakalářský}
%\newcommand\StudProgram{Elektrotechnika a informatika, dobíhající, Magisterský}
% \newcommand\StudProgram{Elektrotechnika a informatika, strukturovaný, Bakalářský}
% \newcommand\StudProgram{Elektrotechnika a informatika, strukturovaný, Navazující magisterský}
 \newcommand\StudProgram{Softwarové technologie a management, Bakalářský}
% English study:
% \newcommand\StudProgram{Electrical Engineering and Information Technology}  % bachelor programe
% \newcommand\StudProgram{Electrical Engineering and Information Technology}  %master program

%%%%%%%%%%%%%%%%%%%%%%%%%%%%%%%%%%%%%
% Zvolte jednu z moznosti 
% Choose one of the following options
%%%%%%%%%%%%%%%%%%%%%%%%%%%%%%%%%%%%%
% nabidky jsou z: http://www.fel.cvut.cz/cz/education/bk/prehled.html

%\newcommand\StudBranch{Výpočetní technika}   % pro program EaI bak. (dobihajici i strukt.)
%\newcommand\StudBranch{Výpočetní technika}   % pro prgoram EaI mag. (dobihajici i strukt.)
\newcommand\StudBranch{Softwarové inženýrství}            %pro STM
%\newcommand\StudBranch{Web a multimedia}                  % pro STM
%\newcommand\StudBranch{Computer Engineering}              % bachelor programe
%\newcommand\StudBranch{Computer Science and Engineering}  % master programe


%%%%%%%%%%%%%%%%%%%%%%%%%%%%%%%%%%%%%%%%%%%%
% Vyplnte nazev prace, autora a vedouciho
% Set up Work Title, Author and Supervisor
%%%%%%%%%%%%%%%%%%%%%%%%%%%%%%%%%%%%%%%%%%%%

\newcommand\WorkTitle{Systém pro evidenci hospitací v Ruby on Rails}
\newcommand\FirstandFamilyName{Tomáš Turek}
\newcommand\Supervisor{Ing. Martin Komárek}


% Pouzijete-li pdflatex, tak je prijemne, kdyz bude mit vase prace
% funkcni odkazy i v pdf formatu
\usepackage[
pdftitle={\WorkTitle},
pdfauthor={\FirstandFamilyName},
bookmarks=true,
colorlinks=true,
breaklinks=true,
urlcolor=red,
citecolor=blue,
linkcolor=blue,
unicode=true,
]
{hyperref}



% Extension posted by Petr Dlouhy in order for better sources reference (\cite{} command) especially in Czech.
% April 2010
% See comment over \thebibliography command for details.

\usepackage[square, numbers]{natbib}             % sazba pouzite literatury
%\usepackage{url}
%\DeclareUrlCommand\url{\def\UrlLeft{<}\def\UrlRight{>}\urlstyle{tt}}  %rm/sf/tt
%\renewcommand{\emph}[1]{\textsl{#1}}    % melo by byt kurziva nebo sklonene,
\let\oldUrl\url
\renewcommand\url[1]{<\texttt{\oldUrl{#1}}>}




\begin{document}

%%%%%%%%%%%%%%%%%%%%%%%%%%%%%%%%%%%%%
% Zvolte jednu z moznosti 
% Choose one of the following options
%%%%%%%%%%%%%%%%%%%%%%%%%%%%%%%%%%%%%
\selectlanguage{czech}
%\selectlanguage{english} 

% prikaz \typeout vypise vyse uvedena nastaveni v prikazovem okne
% pro pohodlne ladeni prace


\iflanguage{czech}{
	 \typeout{************************************************}
	 \typeout{Zvoleny jazyk: cestina}
	 \typeout{Typ prace: \TypeOfWork}
	 \typeout{Studijni program: \StudProgram}
	 \typeout{Obor: \StudBranch}
	 \typeout{Jmeno: \FirstandFamilyName}
	 \typeout{Nazev prace: \WorkTitle}
	 \typeout{Vedouci prace: \Supervisor}
	 \typeout{***************************************************}
	 \newcommand\Department{Katedra počítačů}
	 \newcommand\Faculty{Fakulta elektrotechnická}
	 \newcommand\University{České vysoké učení technické v Praze}
	 \newcommand\labelSupervisor{Vedoucí práce}
	 \newcommand\labelStudProgram{Studijní program}
	 \newcommand\labelStudBranch{Obor}
}{
	 \typeout{************************************************}
	 \typeout{Language: english}
	 \typeout{Type of Work: \TypeOfWork}
	 \typeout{Study Program: \StudProgram}
	 \typeout{Study Branch: \StudBranch}
	 \typeout{Author: \FirstandFamilyName}
	 \typeout{Title: \WorkTitle}
	 \typeout{Supervisor: \Supervisor}
	 \typeout{***************************************************}
	 \newcommand\Department{Department of Computer Science and Engineering}
	 \newcommand\Faculty{Faculty of Electrical Engineering}
	 \newcommand\University{Czech Technical University in Prague}
	 \newcommand\labelSupervisor{Supervisor}
	 \newcommand\labelStudProgram{Study Programme} 
	 \newcommand\labelStudBranch{Field of Study}
}




%%%%%%%%%%%%%%%%%%%%%%%%%%    Poznamky ke kompletaci prace
% Nasledujici pasaz uzavrenou v {} ve sve praci samozrejme 
% zakomentujte nebo odstrante. 
% Ve vysledne svazane praci bude nahrazena skutecnym 
% oficialnim zadanim vasi prace.
{
\pagenumbering{roman} \cleardoublepage \thispagestyle{empty}
\chapter*{Systém pro evidenci hospitací v Ruby on Rails}
Rozšiřte prototyp aplikace pro evidenci hospitací (http://kvalitavyuky.felk.cvut.cz/) tak, aby jej bylo možné nasadit do reálného provozu na ČVUT FEL. Systém implementujte na platformě Ruby on Rail. Vývoj provádějte iterativním způsobem a postupně zapracovávejte požadavky zadavatele. Celý vývoj řádně dokumentujte a výsledky své práce otestujte. Funkčnost systému demonstrujte na hospitacích, které budou probíhat v programu STM v LS 2011/2012.
\newpage
}

%%%%%%%%%%%%%%%%%%%%%%%%%%    Titulni stranka / Title page 

\coverpagestarts

%%%%%%%%%%%%%%%%%%%%%%%%%%%    Podekovani / Acknowledgements 

\acknowledgements
\noindent
Zde můžete napsat své poděkování, pokud chcete a máte komu děkovat.


%%%%%%%%%%%%%%%%%%%%%%%%%%%   Prohlaseni / Declaration 

\declaration{V~České Lípě dne 5.\,2.\,2012}
%\declaration{In Kořenovice nad Bečvárkou on May 15, 2008}


%%%%%%%%%%%%%%%%%%%%%%%%%%%%    Abstract 
 
\abstractpage
\noindent
The purpose of this work is to design and develop a system for evidence inspections of classes. This application will be used at the Faculty of Electrical Engineering. Reason to create the system is to relieve the administrative burden that increases with increase in the number of observations. The application is based on the existing processes of management inspections for the study program STM. The application is created on the platform of Ruby on Rails and builds on a prototype application. The real development of the application is headed by an iterative method of software development. The resulting application is deployed at https://kvalitavyuky.felk.cvut.cz. I have demonstrated the functionality of the on-going observation in the summer semester 2011/2012.

% Prace v cestine musi krome abstraktu v anglictine obsahovat i
% abstrakt v cestine.
\vglue60mm

\noindent{\Huge \textbf{Abstrakt}}
\vskip 2.75\baselineskip

\noindent
Účelem této práce je navrhnout a vytvořit systém pro evidenci hospitací. Tato aplikace bude používaná na Fakultě elektrotechnické. Důvodem vzniku systému je ulehčit administrativní zátěž, která se zvyšuje s nárůstem počtu hospitací. Aplikace je založena na již existujících procesech správy hospitací pro studijní obor STM. Aplikace je vytvořená na platformě Ruby on Rails a navazuje na prototyp aplikace. Samotný vývoj aplikace je veden iterativní metodou vývoje softwaru. Výsledná aplikace je nasazená na adrese https://kvalitavyuky.felk.cvut.cz. Funkčnost jsem demonstroval na probíhajících hospitací v letním semestru 2011/2012.

%%%%%%%%%%%%%%%%%%%%%%%%%%%%%%%%  Obsah / Table of Contents 

\tableofcontents


%%%%%%%%%%%%%%%%%%%%%%%%%%%%%%%  Seznam obrazku / List of Figures 

\listoffigures


%%%%%%%%%%%%%%%%%%%%%%%%%%%%%%%  Seznam tabulek / List of Tables

\listoftables


%**************************************************************

\mainbodystarts
% horizontalní mezera mezi dvema odstavci
%\parskip=5pt
%11.12.2008 parskip + tolerance
\normalfont
\parskip=0.2\baselineskip plus 0.2\baselineskip minus 0.1\baselineskip

% Odsazeni prvniho radku odstavce resi class book (neaplikuje se na prvni 
% odstavce kapitol, sekci, podsekci atd.) Viz usepackage{indentfirst}.
% Chcete-li selektivne zamezit odsazeni 1. radku nektereho odstavce,
% pouzijte prikaz \noindent.

%**************************************************************

% Pro snadnejsi praci s vetsimi texty je rozumne tyto rozdelit
% do samostatnych souboru nejlepe dle kapitol a tyto potom vkladat
% pomoci prikazu \include{jmeno_souboru.tex} nebo \include{jmeno_souboru}.
% Napr.:
% \chapter{Úvod}

Na FEL ČVUT byl zaveden pro studijní program STM (Softwarové technologie a management) systém pro ověřování kvality výuky. Cílem tohoto systému je zkvalitnit vyučované předměty. Jedním ze zdrojů informací jsou kontrolní návštěvy ve výukách\footnote{zkráceně hospitace}. Účelem těchto návštěv je získání komplexního obrazu o kvalitě výuky pro garanty jednotlivých předmětů. A zároveň slouží pro pedagogy jako zpětná vazba z přednášek a cvičení.

Mým cílem mé práce je navrhnout a vytvořit informační systém pro správu hospitací, který zjednoduší a zrychlí administrativu, ke které se doposud používala e-mailová komunikace a www stránky Rady programu \cite{kvalitavyukyweb}. Systém je postaven frameworku Ruby on Rails, který je určen pro vývoj moderních webových aplikací.

% \chapter{Popis problému, specifikace cíle}
\section{Motivace}
Jak jsem uvedl v úvodu, tak v současné době probíhá jakákoliv administrativní činnost okolo hospitací převážně pomocí emailové komunikace. Kterou zajišťuje garantem studijního oboru přidělený administrátor kontroly výuky, který je dále  uváděn jako administrátor, nebo administrátor hospitací. 

Jeho úkolem je starat se o plánování hospitací a vystavování dokumentů na stránkách rady studijního programu. Při naplánování hospitace musí administrátor ručně obeslat emailem informaci o naplánované hospitaci všem zainteresovaným osobám. To jsou hospitovaní, hospitující, přednášející a garanti příslušného předmětu. Po provedení hospitace je potřeba shromáždit a vystavit veškeré dokumenty na webových stránkách. Administrátor musí hlídat tok dokumentů a rozesílat vzniklé dokumenty mezi účastníky hospitace.

Tento systém sice funguje, ale je administrativně a časově náročný jak pro administrátora hospitací, který se stará o komunikaci mezi účastníky, tak i pro zúčastněné strany hospitace. Proto byl podán požadavek na vytvoření systému pro evidenci hospitací, který zautomatizuje vnitřní procesy pro správu hospitací.

\section{Cíle práce}
Hlavním cílem mé práce je rozšířit prototyp aplikace pro evidenci hospitací, tak aby bylo možné ji nasadit do reálného provozu na FEL ČVUT. V průběhu letního semestru 2011/2012 se bude postupně demonstrovat její funkčnost na realizovaných hospitacích v daném semestru. Proto je vývoj veden pomocí iterativního způsobu.

\section{Rešerše}
Hospitace jak už jsem nastínil v motivaci je zavedený vnitřní proces kontroly kvality výuky na ČVUT FEL, proto čerpám informace pro tvorbu této práce z dvou hlavních zdrojů. Prvním zdrojem je směrnice Postupy pro kontrolu kvality výuky a druhým zdrojem je prototyp aplikace z rozpracované bakalářské práce Daniela Krężeloka Návrh a implementace systému pro správu hospitací.

\subsection{Prototyp}
Z prototypu aplikace čerpám způsob propojení aplikace s KOS prostřednictvím webové RESTful služby KOSapi. Tento prototyp je napsán ve frameworku Ruby on Rails a používá knihovnu ze školního projektu VyVy pro komunikaci se službou KOSapi.

\subsection{Plánování hospitací}
Pro každý studijní obor je přiřazen administrátor kontroly kvality výuky. Tato osoba má za úkol plánovat hospitace pro předměty studijního programu. Ten naplánuje hospitaci a přidělí k ní typicky dvojce hospitujících pedagogů. Hospitace může probíhat jak na přednášky, tak i na cvičení. Hospitace jsou tři druhů:

\begin{itemize}
\item Předem ohlášené na konkrétní datum - u tohoto typu hospitace je potřeba, aby administrátor naplánoval datum hospitace s předstihem a informoval o tom hospitovaného.
\item Předem ohlášené bez konkrétního termínu - tento typ na rozdíl od předešlého typu nemá pevně stanovené datum hospitace. 
\item Předem neohlášené - tento typ hospitace se předen neohlašuje hospitovanému. 
\end{itemize}

\subsection{Provedení hospitace}
Při vykonávání hospitace je výstupem písemný zápis, který slouží k popisu průběhu výuky. Tato dokumentační část se píše ručně při hospitaci a dokument předá hospitující administrátorovi kontroly kvality, ten jej odešle hospitovanému a vystaví dokument na privátní části webových stránek Rady programu.

\subsection{Hodnocení výuky}
Hodnotící část se skládá ze tří dokumentů vyplněných hospitujícími a jednoho dokumentu vyplněného hospitovaným. Po sepsání všech dokumentů je hospitace ukončená. Jsou to dokumenty:
\begin{itemize}
\item[A] Hodnocení výuky při hospitaci - ten slouží k ručnímu vyplnění při hospitaci. Skládá se z dokumentační části průběhu hospitace a z hodnotící části. Tento formulář vyplňují hospitující.
\item[B] Slovní hodnocení hospitační návštěvy hospitujícím(mi - jedná se o nejdůležitější část hodnocení, kde jeden z hospitujících sepíše slovní hodnocení.
\item[C] Stanovisko hodnoceného učitele k názorům hospitujícího - je formulář, který slouží hospitovanému k vyjádření o hospitaci.
\item[D] Závěrečné shrnutí hospitujícím - je poslední dokument. Sepisuje ho hospitující a tento dokument obsahuje klady, zápory, navržená opatření a závěr. Tento dokument je pak vyvěšen na veřejných stránkách.
\end{itemize}
% atd...

%*****************************************************************************
\chapter{Úvod}

Na FEL ČVUT byl zaveden pro studijní program STM (Softwarové technologie a management) systém pro ověřování kvality výuky. Cílem tohoto systému je zkvalitnit vyučované předměty. Jedním ze zdrojů informací jsou kontrolní návštěvy ve výukách\footnote{zkráceně hospitace}. Účelem těchto návštěv je získání komplexního obrazu o kvalitě výuky pro garanty jednotlivých předmětů. A zároveň slouží pro pedagogy jako zpětná vazba z přednášek a cvičení.

Mým cílem mé práce je navrhnout a vytvořit informační systém pro správu hospitací, který zjednoduší a zrychlí administrativu, ke které se doposud používala e-mailová komunikace a www stránky Rady programu \cite{kvalitavyukyweb}. Systém je postaven frameworku Ruby on Rails, který je určen pro vývoj moderních webových aplikací.


%*****************************************************************************
\chapter{Popis problému, specifikace cíle}
\section{Motivace}
Jak jsem uvedl v úvodu, tak v současné době probíhá jakákoliv administrativní činnost okolo hospitací převážně pomocí emailové komunikace. Kterou zajišťuje garantem studijního oboru přidělený administrátor kontroly výuky, který je dále  uváděn jako administrátor, nebo administrátor hospitací. 

Jeho úkolem je starat se o plánování hospitací a vystavování dokumentů na stránkách rady studijního programu. Při naplánování hospitace musí administrátor ručně obeslat emailem informaci o naplánované hospitaci všem zainteresovaným osobám. To jsou hospitovaní, hospitující, přednášející a garanti příslušného předmětu. Po provedení hospitace je potřeba shromáždit a vystavit veškeré dokumenty na webových stránkách. Administrátor musí hlídat tok dokumentů a rozesílat vzniklé dokumenty mezi účastníky hospitace.

Tento systém sice funguje, ale je administrativně a časově náročný jak pro administrátora hospitací, který se stará o komunikaci mezi účastníky, tak i pro zúčastněné strany hospitace. Proto byl podán požadavek na vytvoření systému pro evidenci hospitací, který zautomatizuje vnitřní procesy pro správu hospitací.

\section{Cíle práce}
Hlavním cílem mé práce je rozšířit prototyp aplikace pro evidenci hospitací, tak aby bylo možné ji nasadit do reálného provozu na FEL ČVUT. V průběhu letního semestru 2011/2012 se bude postupně demonstrovat její funkčnost na realizovaných hospitacích v daném semestru. Proto je vývoj veden pomocí iterativního způsobu.

\section{Rešerše}
Hospitace jak už jsem nastínil v motivaci je zavedený vnitřní proces kontroly kvality výuky na ČVUT FEL, proto čerpám informace pro tvorbu této práce z dvou hlavních zdrojů. Prvním zdrojem je směrnice Postupy pro kontrolu kvality výuky a druhým zdrojem je prototyp aplikace z rozpracované bakalářské práce Daniela Krężeloka Návrh a implementace systému pro správu hospitací.

\subsection{Prototyp}
Z prototypu aplikace čerpám způsob propojení aplikace s KOS prostřednictvím webové RESTful služby KOSapi. Tento prototyp je napsán ve frameworku Ruby on Rails a používá knihovnu ze školního projektu VyVy pro komunikaci se službou KOSapi.

\subsection{Plánování hospitací}
Pro každý studijní obor je přiřazen administrátor kontroly kvality výuky. Tato osoba má za úkol plánovat hospitace pro předměty studijního programu. Ten naplánuje hospitaci a přidělí k ní typicky dvojce hospitujících pedagogů. Hospitace může probíhat jak na přednášky, tak i na cvičení. Hospitace jsou tři druhů:

\begin{itemize}
\item Předem ohlášené na konkrétní datum - u tohoto typu hospitace je potřeba, aby administrátor naplánoval datum hospitace s předstihem a informoval o tom hospitovaného.
\item Předem ohlášené bez konkrétního termínu - tento typ na rozdíl od předešlého typu nemá pevně stanovené datum hospitace. 
\item Předem neohlášené - tento typ hospitace se předen neohlašuje hospitovanému. 
\end{itemize}

\subsection{Provedení hospitace}
Při vykonávání hospitace je výstupem písemný zápis, který slouží k popisu průběhu výuky. Tato dokumentační část se píše ručně při hospitaci a dokument předá hospitující administrátorovi kontroly kvality, ten jej odešle hospitovanému a vystaví dokument na privátní části webových stránek Rady programu.

\subsection{Hodnocení výuky}
Hodnotící část se skládá ze tří dokumentů vyplněných hospitujícími a jednoho dokumentu vyplněného hospitovaným. Po sepsání všech dokumentů je hospitace ukončená. Jsou to dokumenty:
\begin{itemize}
\item[A] Hodnocení výuky při hospitaci - ten slouží k ručnímu vyplnění při hospitaci. Skládá se z dokumentační části průběhu hospitace a z hodnotící části. Tento formulář vyplňují hospitující.
\item[B] Slovní hodnocení hospitační návštěvy hospitujícím(mi - jedná se o nejdůležitější část hodnocení, kde jeden z hospitujících sepíše slovní hodnocení.
\item[C] Stanovisko hodnoceného učitele k názorům hospitujícího - je formulář, který slouží hospitovanému k vyjádření o hospitaci.
\item[D] Závěrečné shrnutí hospitujícím - je poslední dokument. Sepisuje ho hospitující a tento dokument obsahuje klady, zápory, navržená opatření a závěr. Tento dokument je pak vyvěšen na veřejných stránkách.
\end{itemize}

%*****************************************************************************
\chapter{Analýza}
Tato kapitola pojednává o analýze a návrhu vhodného řešení aplikace. Výstupem této analýzy jsou funkční a obecné požadavky. Dále návrh a popis domén a nejdůležitější případy užití s aktéry. 

\section{Požadavky}
Požadavky na systém se dělí na dvě sekce: obecné a funkční požadavky. Pro definování těchto požadavků jsem vycházel z oficiálního zadání práce tak i z prototypu aplikace, protože mi přesně definuje návrh aplikace a pro implementaci systému i potřebné technologie.

\subsection{Obecné požadavky}
Obecné požadavky se netýkají funkčnosti, ale celkového návrhu a použitých technologií.
\begin{enumerate}
\item Systém bude postaven na webovém frameworku Ruby on Rails.
\item Systém bude webovou aplikací.
\item Systém bude používat webovou službu KOSapi.
\item Systém bude pro autentizaci používat FELid.
\end{enumerate}

\subsection{Funkční požadavky}
Tato sekce se zabývá požadavky na funkčnost systému.
\begin{enumerate}
\item Systém umožní spravovat uživatele.
\item Systém umožní průběžné plánování hospitaci.
\item Systém umožní hospitujícímu i hospitovanému prohlížet hospitace \cite{prototyp_documentace}.
\item Systém umožní vystavit závěrečné hodnocení na veřejné části aplikace \cite{prototyp_documentace}. 
\item Systém umožní hospitovanému sepsat stanoviska k názorům hospitujícího \cite{prototyp_documentace}.
\item Systém umožní hospitujícímu nahrát naskenovaný dokument hodnocení výuky \cite{prototyp_documentace}.
\item Systém umožní hospitujícímu napsat slovní hodnocení z výuky \cite{prototyp_documentace}.
\item Systém umožní hospitujícímu napsat závěrečné shrnutí hospitace \cite{prototyp_documentace}.
\item Systém bude odesílat emailem zprávy o vyplnění hodnotícího dokumentu příslušným osobám \cite{prototyp_documentace}.
\item Systém umožní vyhledávat předměty z KOSapi.
\item Systém umožní vyhledávat osoby z KOSapi.
\item Systém umožní upravovat strukturu hodnotících dokumentů.
\item Systém umožní spravovat emailové šablony k hodnotících dokumentů.
\item Systém umožní generovat emailové zprávy ze šablon.
\item Systém bude automaticky zálohovat databázi.
\end{enumerate}

\newpage 
\section{Uživatelské role}
V systému je celkem 7 uživatelských rolí. Definoval jsem tři základní uživatelské role, které jsou základem systému: nepřihlášený uživatel, přihlášený uživatel, administrátor hospitací a admin. Další dvě role hospitovaný a hospitující se přidělují v rámci jednotlivých hospitací. Na obrázku \ref{fig:actors} jsou vidět jednotlivý aktéři a jejich zobecnění. V další části této sekce rozeberu jednotlivé role a k ním případy užití.

\paragraph*{Use cases}
neboli případy užití je nástroj pro popsání chování jak by systém měl spolupracovat s koncovým uživatelem. Popisuje všechny způsoby jak uživatel komunikuje se systémem.

\begin{figure}[H]
\begin{center}
\includegraphics[width=10cm]{figures/Actors}
\caption{Aktéři}
\label{fig:actors}
\end{center}
\end{figure}

\subsection{Nepřihlášený uživatel}
Nepřihlášený uživatel je role pro hosty naší aplikace. V systému má ze všech rolí nejmenší pravomoc. V tomto stavu je každý uživatel, který se doposud nepřihlásil do systému.

\begin{figure}[H]
\begin{center}
\includegraphics[width=10cm]{figures/actor_base}
\caption{Use case - nepřihlášený uživatel}
\label{fig:actor_base}
\end{center}
\end{figure}

\subsection{Přihlášený uživatel}
Přihlášený uživatel vychází z role nepřihlášeného uživatele. Je to uživatel, který se do systému přihlásil. Jedná se o základní roli pro všechny další role, které ji rozšiřují.

\begin{figure}[H]
\begin{center}
\includegraphics[width=10cm]{figures/actor_logged}
\caption{Use case - přihlášený uživatel}
\label{fig:actor_logged}
\end{center}
\end{figure}


\subsection{Hospitovaný}
Hospitovaný je role pro přihlášeného uživatele v systému. Je přidělena pro každého vyučujícího, který vyučuje předmět, na němž byla naplánovaná hospitace a proběhla.

\begin{figure}[H]
\begin{center}
\includegraphics[width=10cm]{figures/actor_observed}
\caption{Use case - hospitovaný}
\label{fig:actor_observed}
\end{center}
\end{figure}

\subsection{Hospitující}
Hospitující je role pro přihlášeného uživatele v systému. Tato role se přiděluje automaticky z naplánovaných hospitací, nebo ji může přidělit administrátor hospitací.

\begin{figure}[H]
\begin{center}
\includegraphics[width=10cm]{figures/actor_observer}
\caption{Use case - hospitující}
\label{fig:actor_observer}
\end{center}
\end{figure}

\subsection{Administrátor hospitací}
Hlavním úkolem této role je plánovat hospitace na předměty a posléze je spravovat.

\begin{figure}[H]
\begin{center}
\includegraphics[width=10cm]{figures/actor_admin}
\caption{Use case - administrátor hospitací}
\label{fig:actor_admin}
\end{center}
\end{figure}

\subsection{Administrátor}
Administrátor je super uživatel, který má nejvyšší pravomoc v systému. Má přístup ke všem zdrojům aplikace a může aplikaci spravovat.

\begin{figure}[H]
\begin{center}
\includegraphics[width=10cm]{figures/actor_root}
\caption{Use case - administrátor}
\label{fig:actor_root}
\end{center}
\end{figure}

\newpage 
\section{Doménový model}
Doménový model na obrázku \ref{fig:domainmodel} reprezentuje entity v systému a jejich vzájemné vztahy. 

Popis domény jsem pro přehlednost rozdělil podle zdroje na dvě základní skupiny. V první skupině jsou domény, které jsem převzal ze struktury KOSapi a druhou skupinou jsou domény specifické pro moji aplikaci. 

\label{sec:domeny_kosapi} 
\subsection{Domény z KOSapi}
\begin{itemize}
\item Osoba - je osoba v KOSu. Každá osoba může být učitelem a studentem.
\item Semestr - semestr vyučovaný na FEL. 
\item Předmět - předměty vyučované na FEL.
\item Instance předmětu - jsou instance předmětu vypsané v konkrétním semestru.
\item Paralelka - je vypsaná rozvrhová paralelka pro instanci předmětu.
\item Místnost - místnost na FEL
\end{itemize}

\subsection{Domény aplikace}
\begin{itemize}
\item Hospitace - obsahuje informace o naplánování hospitace. 
\item Poznámka - je textová poznámka k plánování hospitace.
\item Hodnocení - reprezentuje informace z proběhlé hospitace hospitace, jsou to informace o datu vykonání hospitace, hospitujícím, předmětu a garantovi.
\item Příloha - je připojený datový soubor k hodnocení hospitace.
\item Šablona formuláře - šablona pro tvorbu formulářů. Definuje vlastnosti jakým se budou vytvářet hodnotící formuláře.
\item Položka - položka reprezentuje jednotlivé segmenty formuláře. Tyto segmenty pak v celku definují strukturu formuláře.
\item Formulář - vyplněný hodnotící formulář.
\item Hodnota - je hodnota z vyplněného formuláře. Ta se ukládá z položky formuláře.
\item Šablona emailu - šablona emailu ze které se budou generovat emailové zprávy.
\end{itemize}

\begin{figure}[p]
\begin{center}
\includegraphics[width=14cm]{figures/DomainModel2}
\caption{Doménový model}
\label{fig:domainmodel}
\end{center}
\end{figure}



\section{Životní cyklus hospitace}
Cílem této části analýzy je popsat životní cyklus, kterým hospitace prochází. 

\subsection{Vytvoření}
Životní cyklus hospitace začíná jejím vytvořením. Toto zajišťuje administrátor hospitací, který založí hospitaci a definuje semestr, kdy se má hospitace uskutečnit, a předmět vyučovaný na fakultě. Při vytváření hospitace se určí typ hospitace a tím i její způsob zviditelnění, pro ostatní aktéry v aplikaci.

\subsection{Naplánování}
Při plánování je také hlavním aktérem administrátor hospitace. V této části životního cyklu administrátor určí hospitovanou paralelku předmětu a datum,  kdy se hospitace uskuteční.  

Administrátor také v této fázi přidělí hospitující z řad pedagogů určených k vykonání hospitace.  
 
\subsection{Hodnocení}
Poté, co proběhla kontrola hospitace, začíná nová fáze, ve které se hodnotí vyučování. Do této fáze už nezasahuje administrátor hospitace, ale přicházejí na scénu dva jiní aktéři: hospitovaný a hospitující.

V první fázi musí hospitující vyplnit, nebo nahrát naskenovaný formulář pro Hodnocení výuky při hospitaci. Tento formulář slouží k dokumentaci průběhu hospitace.

V druhé fázi jeden z hospitujících sepíše slovní hodnocení hospitační návštěvy.

Ve třetí fázi může hospitovaný do dvou dnů vyplnit stanovisko hodnoceného k názorům hospitujícího.  

V poslední fázi jeden z hospitujících sepíše poslední formulář Závěrečné shrnutí. Po vyplnění tohoto formuláře se hospitace stává ukončenou a tím končí její životní cyklu.
 
\subsection{Ukončená}
Po sepsáním posledního hodnotícího dokumentu se hospitace dostane do fáze ukončená. 

\begin{figure}[h]
\begin{center}
\includegraphics[width=14cm]{figures/hospitace}
\caption{Život hospitace}
\label{fig:hospitace}
\end{center}
\end{figure}

\chapter{Návrh}

\section{Technologie a služby}
Tato část popisuje jednotlivé technologie a služby potřebné pro implementaci aplikace.

\subsection{Ruby on Rails}
Ruby on Rails \cite{rubyonrails}, zkráceně Rails, je jedno z implementačních omezení, které se nachází přímo v zadání práce. Jedná se o framework primárně určený pro vývoj webových aplikací napojených na relační databázi. Framework je napsán na skriptovacím interpretovaným programovací jazyku Ruby \cite{ruby}. Rails používá návrhový vzor Model-view-controller viz. \ref{mvc}. Mezi základními myšlenkami a filozofií frameworku patří dva principy. Prvním principem je Convention over Configuration viz. \ref{coc} a druhým je Don’t Repeat Yourself viz. \ref{dry}.

\subsubsection{Balíckovací systém RubyGems}
Protože jsou Rails postaveny na jazyku Ruby lze používat balíčkovací systém RubyGems, kde existuje velké množství již existujících řešení, které jsou dostupné jako knihovny\footnote{v RubyGems se knihovna nazývá Gem}. 

\subsection{KOSapi}
\label{kosapi}
KOSapi je webová služba poskytující aplikační rozhraní v podobě RESTful webové služby \ref{rest}. Je určená pro vznik školních aplikací, které potřebují mít přístup k datům souvisejících s výukou. Pro aplikaci používám stabilní verzi API 2. 

Z této služby čerpám hlavně data předmětů a osob v KOSu. Pro připojení ke KOSapi používám již existující knihovnu napsanou v Ruby Tomášem Linhartem a Tomášem Jukínem ve školním projektu VyVy \cite{vyvy_project}.  

\subsection{FELid}
\label{felid}
FELid \cite{felid} je globální autentizační a autorizační systém pro webovské aplikace na síti FEL. Poskytuje jednotný a bezpečný způsob přihlášení uživatelů a přenos jejich údajů do různých aplikací na webu. Zároveň podporuje jednorázové přihlášení (tzv. single sign-on). Znamená to, že se uživatel přihlašuje pouze do první použité aplikaci a u dalších aplikací už nemusí zadávat svoje přihlašovací údaje.

Tuto službu používám pro autentizaci uživatelů do systému. Abych mohl používat v aplikaci FELid je nutné splnit technické požadavky, které jsou napsány na stránkách FELid \cite{felid_pozadavky}.

\subsection{Aplikační server}
\label{apache}
Pro zprovoznění aplikace do reálného provozu jsem použil webový server Apache HTTP server \cite{apache} ve verzi 2. Tento aplikační server jsem zvolil kvůli obecným požadavkům aplikace pro využití FELid a pažadavku aby aplikace byla napsaná v Ruby on Rails. Tato verze webového serveru totiž umožňuje instalaci zásuvných modulů Passenger \cite{passenger} a Shibboleth \cite{shibboleth}. Passenger umožňuje nasazení rails aplikací na aplikačním serveru. Druhý modul Shibboleth zprostředkovává single sign-on autentizaci mezi aplikačním serverem a službou FELid.

\subsection{Databáze}
Ruby on Rails poskytuje možnost připojení k různým databázovým systémům prostřednictvím adaptérů. Díky tomu není aplikace závislá na použitém databázovém systému a díky tomu mohu používat pro vývoj a testování aplikace jednoduchý databázový systém SQLite \cite{sqlite}, který pro tyto účely bohatě postačuje a není potřeba jej složitě konfigurovat. Pro samotné nasazení aplikace do provozu už používám databázový systém MySQL \cite{mysql}.

\section{Architektura}
V této části popisuji použité architektonické vzory a konvence, které dodržuji při návrhu aplikace.

\subsection{MVC}
\label{mvc}
MVC (Model-view-controller) \cite{mvc} je softwarová architektura, která rozděluje datový model aplikace, uživatelské rozhraní a řídicí logiku do tří nezávislých komponent\footnote{models, views a controllers} tak, že modifikace některé z nich má minimální vliv na ostatní. Tento architektonický vzor obsahuje ve svém jádru Ruby on Rails, proto pro implementaci tohoto vzoru vycházím z fungování frameworku.

\subsubsection{Models}
Model reprezentuje informace v aplikaci a pravidla pro práci s nimi. V případě Rails jsou modely primárně využívány pro interakci s příslušnou tabulkou v databázi a pro ukládání pravidel této interakce. Ve většině případů odpovídá jedna tabulka v databázi jednomu modelu aplikaci. Modely obsahují většinu aplikační logiky.

\subsubsection{Views}
Pohledy, neboli views reprezentují uživatelské rozhraní aplikace. V Rails jsou views obvykle HTML soubory s vloženými částmi Ruby kódu, který provádí pouze úkony týkající se prezentace dat. Views mají na starosti poskytování dat webovému prohlížeči nebo jinému nástroji, který zasílá vaší aplikaci požadavky.

\subsubsection{Controllers}
Kontrolory fungují jako zprostředkovatel mezi modely a views. V Rails slouží kontrolory k zpracování požadavků které přichází z webového prohlížeče, získávání dat z modelů a k odesílání těchto dat do views, kde budou zobrazeny.

\subsection{DRY}
\label{dry}
DRY (Don’t repeat yourself) \cite{dry} je princip vývoje softwaru zaměřený na snížení opakování psaní stejného kódu a tím zvyšuje čitelnost a znovupoužitelnost kódu. To znamená, že informace se nacházejí na jednoznačném místě. Pro příklad Ruby on Rails získává definici sloupců pro třídu modelu přímo z databáze.  

\subsection{CoC}
\label{coc}
CoC (Convention over Configuration) \cite{coc} je další princip používaný v Rails pro zlepšení čitelnosti a znovupoužitelnosti kódu. Tento princip znamená, že konvence má přednost před konfigurací a to tak, že Rails předpokládá to, co chcete udělat, místo toho, aby vás nutil specifikovat každou drobnost v konfiguraci. 

\subsection{REST}
\label{rest}
REST (Representational State Transfer) \cite{rest} je architektonický vzor pro webové aplikace. Je založen na HTTP protokolu a hlavní myšlenkou je poskytovat přístup ke zdrojům dat. Všechny zdroje jsou identifikovány přes URI. REST definuje čtyři základní metody pro přístup ke zdrojům. Jsou známé pod označením CRUD\footnote{create, retriece, update a delete}. Tyto metody jsou implementovány pomocí odpovídajících metod HTTP protokolu. Jednotlivé metody rozeberu na příkladech pro zdroj \verb|observations|\footnote{zdroj aplikace pro práci s hospitacemi}.

\paragraph*{Create}
je požadavek, který pomocí metody POST vytvoří nový záznam. Příklad dotazu vytvoří novou hospitaci.

\begin{quote}
\begin{verbatim}
POST /observations
\end{verbatim} 
\end{quote}

\paragraph*{Retrieve}
je požadavek pro přístup ke zdrojům. Funguje stejným způsobem jako běžný požadavek na stránku pomocí GET metody. V prvním příklad vrátí seznam všech hospitace. Druhý příklad vrátí podrobnosti hospitace s id 1.

\begin{quote}
\begin{verbatim}
GET /observations
GET /observations/1
\end{verbatim} 
\end{quote}
 
\paragraph*{Update}
je požadavek, pro upravení konkrétního záznamu přes metodu PUT. 

\begin{quote}
\begin{verbatim}
PUT /observations/1
\end{verbatim} 
\end{quote}

\paragraph*{Delete}
je požadavek, který smaže konkrétní záznam pomocí DELETE metody.

\begin{quote}
\begin{verbatim}
DELETE /observations/1
\end{verbatim} 
\end{quote}

\section{Struktura aplikace}
V této sekci popíšu základní strukturu aplikace. Struktura aplikace je vygenerovaná pomocí generátorů v Ruby on Rails a proto nepopíšu celou strukturu, ale pouze části aplikace, které byli nejdůležitější pro vývoj aplikace. Na obrázku \ref{tree:hospitace} je popsaná strukturu složek a k nim v popisku jejich obsah.

\begin{figure}[h]
	\dirtree{%
		.1 Hospitace/.
		.2 app/.
		.3 assets/ \DTcomment{obsahuje obrázky, JavaScript, CSS}.
		.3 controllers/ \DTcomment{controllers aplikace}.
		.3 helpers/ \DTcomment{pomocné funkce pro vytváření pohledů}.
		.3 inputs/ \DTcomment{speciální vstupní položky pro tvorbu formulářů}.
		.3 mailers/ \DTcomment{třídy které generují a odesílají emaily}.
		.3 models/ \DTcomment{modely aplikace}.
		.3 views/ \DTcomment{šablony aplikace}.		
		.2 lib/\DTcomment{rozšíření a moduly pro aplikaci}.
		.3 email\_templates/ \DTcomment{modul pro generování emailů}.
		.3 tasks/ \DTcomment{obsahuje rake scripty}.
		.3 kosapi/ \DTcomment{knihovna pro připojení ke KOSapi}.
		.3 will\_paginate/ \DTcomment{rozšíření pro modul will\_paginate}.
		.2 public/ \DTcomment{složka, která je přístupná přes web. Obsahuje statické soubory}.
		.2 config/ \DTcomment{konfigurační soubory a směrování}.		
		.2 db/ \DTcomment{schéma databáze a databázové migrace}. 
		.2 test/\DTcomment{testy, testovací data a nástroje pro testování aplikace}.
	}
	\caption{Struktura aplikace}
\label{tree:hospitace}
\end{figure}
%*****************************************************************************
\chapter{Realizace}
Popis implementace/realizace se zaměřením na nestandardní části řešení.


%*****************************************************************************
\chapter{Testování}

Cílem testování bylo otestovat reálnou funkčnost aplikace a odhalit chyby při návrhu a vývoji aplikace. Aplikaci jsem testoval dvěma způsoby: manuálně, kde jsem musel ručně procházet aplikaci a zkoušet její funkčnost, a automaticky pomocí testovacích nástrojů v Ruby on Rails. Aplikaci jsem testoval v průběhu celého vývoje. 

\section{Automatické testování}
Pro automatické testování jsem použil testovací nástroje v Ruby on Rails \cite{RoR_testing}. V Rails se testy dělí do tří kategorii: Unit, Functional a Integration. Jednotlivé kategorie testů, které jsem použil, popíši dále v textu. Testy jsem tvořil jen na kritických částech aplikace, kde byla velká pravděpodobnost vzniku chyb při úpravách funkcionality a návrhu aplikace.

\subsection{Testovací data}
Testovací nástroje potřebují pro svou funkčnost testovací data. Tyto data se nacházejí ve struktuře programu ve složce tests/fixtures. Zde jsou soubory s testovacími daty pro jednotlivé modely. Data jsou uložená ve formátu YAML\footnote{YAML je formát pro serializaci strukturovaných dat}. Před spuštěním testů se tato data
nahrají do testovací databáze. V mém případě je to lokální databáze SQLite. Díky oddělené databázi nemá testování vliv na produkční databázi.

\subsection{Unit testing}
Unit testy slouží k testování samostatných částí programů. V Rails se tyto testy používají hlavně pro testování funkčnosti modelů. Testuje se hlavně validace vstupů a perzistence dat. Tyto testy mi odhalily spoustu chyb, které vznikly při úpravách entit v aplikaci.

\subsection{Functional tests}
Tyto testy testují různé činnosti v jednotlivých controllerech aplikace. Controllery zpracovávají příchozí webové požadavky a nakonec odpovědí vyrendrovanou šablonou \cite{RoR_testing}.

\begin{list}{•}{\textbf{Tento typ testů testuje:}}
\item Byl webový požadavek úspěšný?
\item Byli jsme přesměrováni na správnou stránku?
\item Byli jsme úspěšně přihlášeni?
\item Byl objekt vložen do správné šablony?
\item Byla zobrazena správná hláška uživateli?
\end{list} 

\section{Manuální testování}
Manuálně jsem testoval ty části aplikace, u kterých se špatně vytvářejí  testy, nebo bylo potřeba použít lidský úsudek. Tyto vlastnosti splňuje testování uživatelského rozhraní a testování autorizace. Nejrozsáhlejší testování bylo vždy před koncem každé iterace. Důvodem bylo rozsáhlé testování reálné funkčnosti aplikace. Více v kapitole \ref{sec:sys_test}.

\section{Systémové testování}
\label{sec:sys_test}
Systémové testy jsou hlavní a nejdůležitější součástí testování. Smyslem tohoto testování
je ověřit funkčnost aplikace a zjistit, zda byly pokryty všechny požadavky dané zadavatelem. Nejčastěji tyto testy bývají prováděny  před nasazením do reálného provozu. 

Tento druh testů jsem prováděl vždy před  
koncem každé iterace, než jsem prezentoval svůj postup. Ujišťoval jsem se o stavu nové verze, zda není rozbitá a také jsem tím získal zpětnou vazbu o postupu vývoje. Při systémovém testování aplikace jsem netestoval všechny funkce aplikace, ale pouze ty nejdůležitější. Jsou to scénáře, které simulují běžnou práci uživatele. Při testování funkčnosti jsem simuloval průběh hospitací pomocí reálných dat získaných z probíhajících hospitací v letním semestru 2011/2012.

V příloze dokumentu \ref{test} jsou popsány jednotlivé testovací případy pro systémové testování. Výsledky těchto testů jsou znázorněny v tabulce \ref{tab:test}. V jednotlivých iteracích je znázorněno, zda aplikace testem prošla. Lze z ní také vyčíst i postupný vývoj aplikace. V poslední iteraci byly splněny všechny systémové testy, proto bylo možné aplikaci nasadit.


%Systémové testy - jsou hlavní částí testování z pohledu vývoje. Zde je ověřováno, že aplikace jako celek funguje správně.Testuje se, že správně plní úlohu, pro kterou byla vyvinuta, že vrací správné výstupy, že byly ošetřeny všechny nestandardní situace a v neposlední řadě, že byly pokryty všechny požadavky ze strany zákazníka. Systémové testy obvykle probíhají v několika kolech. Jsou hlášeny nalezené chyby, ty jsou opraveny a v následujících kolech retestovány.


% Musel jsem projít celou aplikaci a otestovat její funkčnost. 
%Při testování funkčnosti jsem simuloval průběh hospitací. K tomu jsem využíval reálná data z probíhajících hospitací v letním semestru 2011/2012.

\newpage
\begin{table}[h]
\begin{center}
\begin{tabular}{|c|c||c|c|c|c|}
\hline 
\textbf{ID} & \textbf{Test} & \textbf{1. iterace} & \textbf{2. iterace} & \textbf{3. iterace} & \textbf{4. iterace} \\
\hline 
1 & Přihlásit se &  &  &  & $\checkmark$ \\ 
\hline 
2 & Odhlásit se &  &  &  & $\checkmark$ \\ 
\hline 
3 & Seznam hospitací & $\checkmark$ & $\checkmark$ & $\checkmark$ & $\checkmark$ \\ 
\hline 
4 & Seznam hodnocení &  &  & $\checkmark$ & $\checkmark$ \\ 
\hline 
5 & Závěrečné hodnocení &  &  & $\checkmark$ & $\checkmark$ \\ 
\hline 
6 & Seznam hospituji & $\checkmark$ & $\checkmark$ & $\checkmark$ & $\checkmark$ \\ 
\hline 
7 & Zahájit hodnocení &  & $\checkmark$ & $\checkmark$ & $\checkmark$ \\ 
\hline 
8 & Formulář A &  & $\checkmark$ & $\checkmark$ & $\checkmark$ \\ 
\hline 
9 & Nahrát přílohu &  &  & $\checkmark$ & $\checkmark$ \\ 
\hline 
10 & Formulář B &  & $\checkmark$ & $\checkmark$ & $\checkmark$ \\ 
\hline 
11 & Formulář C &  & $\checkmark$ & $\checkmark$ & $\checkmark$ \\ 
\hline 
12 & Formulář D &  & $\checkmark$ & $\checkmark$ & $\checkmark$ \\ 
\hline 
13 & Odesílání emailů &  &  &  & $\checkmark$ \\ 
\hline 
14 & Seznam hospitován &  &  & $\checkmark$ & $\checkmark$ \\ 
\hline 
15 & Vytvořit hospitaci & $\checkmark$ & $\checkmark$ & $\checkmark$ & $\checkmark$ \\ 
\hline 
16 & Přiřadit hospitujícího & $\checkmark$ & $\checkmark$ & $\checkmark$ & $\checkmark$ \\ 
\hline 
17 & Naplánovat hospitaci & $\checkmark$ & $\checkmark$ & $\checkmark$ & $\checkmark$ \\ 
\hline 
18 & Napsat poznámku &  & $\checkmark$ & $\checkmark$ & $\checkmark$ \\ 
\hline 
19 & Přiřadit roli & $\checkmark$ & $\checkmark$ & $\checkmark$ & $\checkmark$ \\ 
\hline 
\end{tabular} 
\caption{Výsledky systémového testování}
\label{tab:test}
\end{center}
\end{table}

\paragraph{Struktura tabulky \ref{tab:test}:}
\subparagraph*{ID}
Identifikátor testovacího případu, ten lze podle ID dohledat v příloze dokumentu \ref{test}.
\subparagraph*{Test}
Hlavní cíl testu.
\subparagraph*{Iterace}
Obsahuje výsledky, zda byl test splněn v jednotlivých iterací.




%*****************************************************************************
\chapter{Závěr}

Cílem bakalářské práce bylo vytvořit webovou aplikaci pro evidenci hospitací, která navazuje na dříve vytvořený prototyp. Musí být v takovém stavu, aby ji šlo použít v reálném provozu na ČVUT Fakultě elektrotechnické. Aplikace má totiž pomoct ulehčit administrativní zátěž, která se zvyšuje s nárůstem počtu hospitací. 

Podařilo se mi vytvořit aplikaci, která slouží k tomuhle účelu a je implementována na platformě Ruby on Rails, což je jeden z nejmodernějších frameworků pro vývoj webových aplikací. Pro nasazení do reálného provozu aplikace používám dvě fakultní aplikace. Pro autentizaci používá aplikaci FELid, což je globální autentizační systém pro webové aplikace FEL. Druhou aplikací je KOSapi, ta poskytuje aplikační rozhraní k přístupu dat v KOSu.

Součástí práce bylo i vytvoření dvou rozsáhlých částí aplikace. První částí jsou dynamické formuláře, které umožňují vytvářet formuláře s možností editace struktury a druhou částí aplikace je knihovna, která umožňuje generovat obsah informačních emailů s daty z hospitace pomocí šablony. Výsledná aplikace je nasazená na serveru http://kvalitavyuky.felk.cvut.cz a připravená na použití v reálném provozu.

\section{Osobní přínos}
Protože obsah práce byl obsáhlý, od implementace aplikace, její ho nasazení, až po instalaci serveru, jsem získal spoustu zkušeností. Největším přínosem pro mně je poznání nového a zajímavého jazyka Ruby a moderním  frameworku Ruby on Rails. Při tvorbě jsem narazil i na problémy s externími aplikacemi, které jsem musel řešit, což mě přimělo hlouběji se touto problematikou zabývat.

\section{Možnosti pokračování}
Možností pokračování v práci je několik. V první řadě je potřeba nainstalovat a nastavit emailový server pro odesílání informačních emailových zpráv.

Lze také pokračovat ve vývoji dynamických formulářů. Do aplikace jsem totiž neimplementoval uživatelské prostředí pro nastavování formulářů. Dále by bylo dobré předělat funkční část do samostatné knihovny. Do budoucna je i možnost rozšířit aplikaci o podporu nové verze KOSapi, která je momentálně ve vývoji. Nová verze bude totiž obsahovat data z Bílé knihy\footnote{je to dokument, který obsahuje informace o organizací studijních plánů} a bude možné pak rozšířit aplikaci o podporu studijních programů i s obory.

%*****************************************************************************
% Seznam literatury je v samostatnem souboru reference.bib. Ten
% upravte dle vlastnich potreb, potom zpracujte (a do textu
% zapracujte) pomoci prikazu bibtex a nasledne pdflatex (nebo
% latex). Druhy z nich alespon 2x, aby se poresily odkazy.

% originally following specification for bibliography formating was used
%\bibliographystyle{abbrv}

% Here is an improvment by Petr Dlouhy (April 2010).
% It is mainly for supervisors who expect Czech fomrating rules for references
% Additional feature is live url addresses to sources from your pdf file
% It requires the file csplainnat.bst (included in this sample zipfile).

\bibliographystyle{csplainnat}

%bibliographystyle{plain}
%\bibliographystyle{psc}
{
%JZ: 11.12.2008 Kdo chce mit v techto ukazkovych odkazech take odkaz na CSTeX:
\def\CS{$\cal C\kern-0.1667em\lower.5ex\hbox{$\cal S$}\kern-0.075em $}
\bibliography{reference}
}

% M. Dušek radi:
%\bibliographystyle{alpha}
% kdy citace ma tvar [AutorRok] (napriklad [Cook97]). Sice to asi neni  podle ceske normy (BTW BibTeX stejne neodpovida ceske norme), ale je to nejprehlednejsi.
% 3.5.2009 JZ polemizuje: BibTeX neobvinujte, napiste a poskytnete nam styl (.bst) splnujici citacni normu CSN/ISO.

%*****************************************************************************
%*****************************************************************************
\appendix
\chapter{Seznam použitých zkratek}

\begin{description}
\item[API] Application Programming Interface
\item[CoC] Convention over Configuration
\item[CRUD] Create Retriece Update Delete
\item[CSS] Cascading Style Sheets
\item[ČVUT] České vysoké učení technické v Praze
\item[DRY] Don't repeat yourself
\item[FEL] Fakulta elektrotechnická
\item[i18n] Internationalization and localization
\item[KOS] Komponenta studium
\item[LS] Letní semestr
\item[MVC] Model-view-controller
\item[HTML] HyperText Markup Language
\item[REST] Representational State Transfer
\item[RVM] Ruby Version Manager
\item[SAML] Security Assertion Markup Language
\item[STM] Softwarové technologie a management
\item[URI] Uniform Resource Identifier
\item[VyVy] Vykazování Výuky
\item[WWW] WorldWideWeb
\item[YAML] Ain't Markup Language
\end{description}


%\include{appendix}

\end{document}
