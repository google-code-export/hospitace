\chapter{Popis problému, specifikace cíle}
\section{Motivace}
Jak jsem uvedl v úvodu, tak v současné době probíhá jakákoliv administrativní činnost okolo hospitací převážně pomocí emailové komunikace. Zajišťuje jí garantem studijního oboru přiřazený administrátor kontroly výuky, který je dále  uváděn jako administrátor, nebo administrátor hospitací. 

Jeho úkolem je starat se o plánování hospitací a vystavování dokumentů na stránkách rady studijního programu \cite{kvalitavyukyweb}. Při naplánování hospitace musí administrátor ručně obeslat emailem informaci o naplánované hospitaci všem zainteresovaným osobám. To jsou hospitovaní, hospitující, přednášející a garanti příslušného předmětu. Po provedení hospitace je potřeba shromáždit a vystavit veškeré dokumenty na webových stránkách Rady programu. Administrátor musí hlídat tok dokumentů a rozesílat vzniklé dokumenty mezi účastníky hospitace.

Tento systém sice funguje, ale je administrativně a časově náročný při zvyšujícím se počtu hospitací jak pro administrátora hospitací, který se stará o komunikaci mezi účastníky, tak i pro zúčastněné strany hospitace. Proto byl podán požadavek na vytvoření systému pro evidenci hospitací, který zautomatizuje vnitřní procesy pro správu hospitací.

\section{Cíle práce}
Hlavním cílem mé práce je rozšířit prototyp aplikace pro evidenci hospitací, tak aby bylo možné ji nasadit do reálného provozu na FEL ČVUT. V průběhu letního semestru 2011/2012 se bude postupně demonstrovat její funkčnost na realizovaných hospitacích v daném semestru.

\section{Rešerše}
Hospitace, jak už jsem nastínil v motivaci, je zavedený vnitřní proces kontroly kvality výuky na ČVUT FEL. Informace proto čerpám ze dvou hlavních zdrojů. Prvním zdrojem je dokument Postupy pro kontrolu kvality výuky \cite{postupy}, který definuje jak se mají hospitace provádět. A druhým zdrojem je prototyp aplikace z rozpracované bakalářské práce Návrh a implementace systému pro správu hospitací \cite{prototyp_documentace}.

\subsection{Prototyp}

Jako referenční řešení práce jsem obdržel prototyp aplikace napsaný v Ruby on Rails. Ten se skládá z dokumentace a z rozpracované webové aplikace. Dokumentace k prototypu obsahuje analýzu a návrh aplikace, což  mi pomohlo při vývoji aplikace a také pochopení problematiky hospitací. Práce se samotnou aplikací byla již obtížnější, protože verze, kterou jsem obdržel, byla velmi rozpracovaná a nešla spustit. 

Po zprovoznění aplikace do spustitelného stavu jsem zjistil, že aplikace umí získávat data ze služby KOSapi\footnote{viz. \ref{kosapi}} a plánovat hospitace. Způsob plánování hospitací je v prototypu uživatelsky nepřívětivý. Pro naplánování hospitace musí uživatel vybírat předmět a hospitujícího pomocí roletkových seznamů bez možnosti filtrování. S tisíci záznamy v seznamu je skoro nemožné najít ten správný záznam. V nedostatečném stavu byly i hodnotící formuláře. Prototyp sice obsahuje formuláře, ale nebyla k nim implementována funkcionalita. 

Důvod proč navazuji na rozpracovanou práci je ten, že původní autor se rozhodl kompletně přejít na jinou technologii než je Ruby on Rails. Nakonec, po hlubší analýze prototypu programu, jsem se rozhodl převzít z prototypu aplikace pouze připojení ke KOSapi a zbytek udělat od základů nový. 

\subsection{Postupy pro kontrolu kvality výuky}
V této části kapitoly popisuji hlavní procesy při vykonávání hospitací. Tyto informace čerpám z již zmíněného dokumentu Postupy pro kontrolu kvality výuky \cite{postupy}, kde jsou obsaženy veškeré potřebné informace o průběhu hospitací.

\subsubsection{Administrátor hospitací}
Ke každému studijnímu programu je přidělena určitá osoba, kterou vybírá garant studijního oboru. Tato osoba se stará o hospitace pro studijní obor, ke je přiřazena. Hlavními úkoly administrátora jsou plánování a řízení hospitací.

\subsubsection{Plánování hospitací}
Pro naplánování hospitace musí administrátor hospitací nadefinovat předmět, paralelku a datum hospitace. Hospitace může probíhat jak na přednáškách, tak i na cvičeních. Do plánování patří i přiřazení hospitujících\footnote{typicky je to dvojce pedagogů jeden zkušený a druhý začínající}, kteří provedou kontrolní návštěvu ve výuce.

\subsubsection{Typy hospitací}
Administrátor také určuje typ hospitace. Jsou zavedeny tři druhy, které se především rozlišují z hlediska hospitovaného.  Jsou to:

\paragraph*{Předem ohlášené na konkrétní datum} 
u tohoto typu hospitace jsou veřejné všechny informace o naplánování hospitace. Proto je potřeba, aby administrátor naplánoval hospitaci s předstihem a informoval o tom hospitovaného.

\paragraph*{Předem ohlášené bez konkrétního termínu} 
tento typ hospitace na rozdíl od předešlého typu nemá pevně stanovený datum hospitace. Hospitovaný sice ví o naplánované hospitaci, ale není zveřejněn datum, kdy proběhne.

\paragraph*{Předem neohlášené} 
tento typ hospitace se používá jen zřídka a to u problémových předmětů, nebo při řešení vážných stížností na vyučujícího. Proto u  tohoto typu hospitace je předem informován pouze garant, zástupce garanta a administrátor výuky.

\subsubsection{Provedení hospitace}
Hospitace se vykoná kontrolní návštěvou hospitujících ve výuce, o které sepíší dokument Hodnocení výuky. V něm zdokumentují průběh hodiny a její hodnocení.

\subsubsection{Hodnocení výuky}
Po vykonání hospitace následuje hodnotící fáze. Ta se skládá ze čtyř druhů dokumentů, které zhodnotí proběhlou výuku. Všechny dokumenty musí být předány administrátorovi hospitací. Ten je vystaví na privátní části webových stránek Rady programu \cite{kvalitavyukyweb}. Po sepsání a vystavení všech dokumentů je hospitace ukončená. Dokumenty používané při hodnocení:

\label{sec:formulare}
\begin{itemize}
\item[A] Hodnocení výuky při hospitaci - ten dokument  je písemný výstup z hospitace. Hlavním účelem tohoto dokumentu je slovně popsat průběh výuky. Dokument se skládá z dokumentační části a z hodnotící části. Tento formulář vyplňuje každý hospitující sám za sebe.
\item[B] Slovní hodnocení hospitační návštěvy hospitujícím(mi) - tento dokument sepíše po provedení hospitace jeden z hospitujících a účelem tohoto dokumentu je slovně zhodnotit výuku. 
\item[C] Stanovisko hodnoceného učitele k názorům hospitujícího - tímto dokumentem se může hospitovaný, garant předmětu a vedoucí katedry vyjádřit k slovnímu hodnocení. Proto tento dokument lze sepsat až po vystavení formuláře B.
\item[D] Závěrečné shrnutí hospitujícím - je poslední a nejdůležitější dokument ze všech. Sepíše ho jeden z hospitujících. Tento dokument slouží jako výstup hodnocení hospitace. Dokument obsahuje klady, zápory, navržená opatření a závěr. Je také jediným veřejným dokumentem a proto je potřeba ho vytavit na volně přístupných stránkách.
\end{itemize}