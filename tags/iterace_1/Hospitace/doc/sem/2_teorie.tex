\chapter{Popis problému, specifikace cíle}
\section{Cíle práce}
Hlavním cílem semestrální práce je rozšířit prototyp aplikace pro evidenci hospitací na platformě Ruby on Rails, tak aby bylo možné nasadit částečně do reálného provozu v letním semestru 2011/2012 pro program STM. Proto je vývoj veden pomocí iterativního způsobu. Součástí semestrální práce je nultá a první iterace.

Momentální stav plánování hospitací na FEL ČVUT je založena na komunikací pomocí systému wiki stránek. Tento způsob evidence je příliš náročný na administrativu a v současné době nevyhovuje. Proto je potřeba vytvořit informační systém, který zautomatizuje vnitřní procesy plánování hospitací. 

\subsection{Nultá iterace}
Součástí nulté iterace bylo seznámení se s procesy plánování hospitací a vytvoření prototypu aplikace, který dokáže komunikovat s webovou službou KOSapi. Pro komunikaci slouží modul převzatý ze školního projektu VyVy. 
Analytickou část semestrálního projektu vycházím z rozpracované bakalářské práce Daniela Krężeloka Návrh a implementace systému pro správu hospitací.
 
\subsection{První iterace}
Součástí první iterace je návrh a implementace procesu naplánování hospitací. 

\section{Rešerše}
Pro zjištění fungování hospitací 

\section{Požadavky}
\subsection{Obecné požadavky}
Obecné požadavky jsou požadavky, které se netýkají funkčnosti ale celkového návrhu a použitých technologií.
\begin{enumerate}
\item Aplikace bude postavena na webovém frameworku Ruby on Rails.
\item Aplikace bude používat databázi MySQL.
\item Serverová část aplikace poběží na aplikačním serveru Apache2
\item Aplikace bude propojená s webovou službou KOSapi.
\item Autentizace do systému pomocí FELid.
\end{enumerate}

\subsection{Funkční požadavky}
Tato sekce se zabývá požadavky na funkčnost systému. Systém umožní:
\begin{enumerate}
\item spravovat uživatele
\item průběžné plánování hospitaci
\item přiřazení hospitujícího k hospitaci
\item vypsat naplánované hospitace
\item najít předmět z KOSapi
\end{enumerate}